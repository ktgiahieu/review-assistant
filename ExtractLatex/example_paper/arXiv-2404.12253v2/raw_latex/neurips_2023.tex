\documentclass{article}


% if you need to pass options to natbib, use, e.g.:
    \PassOptionsToPackage{numbers, compress}{natbib}
% before loading neurips_2023


% ready for submission
% \usepackage{neurips_2023}
\usepackage[final]{neurips_2024}


% to compile a preprint version, e.g., for submission to arXiv, add add the
% [preprint] option:
    % \usepackage[preprint]{neurips_2023}


% to compile a camera-ready version, add the [final] option, e.g.:
%     \usepackage[final]{neurips_2023}

% \usepackage[final]{neurips_2023}

% to avoid loading the natbib package, add option nonatbib:
%    \usepackage[nonatbib]{neurips_2023}

\usepackage[utf8]{inputenc} % allow utf-8 input
\usepackage[T1]{fontenc}    % use 8-bit T1 fonts
\usepackage{hyperref}       % hyperlinks
\usepackage{url}            % simple URL typesetting
\usepackage{booktabs}       % professional-quality tables
\usepackage{amsfonts}       % blackboard math symbols
\usepackage{nicefrac}       % compact symbols for 1/2, etc.
\usepackage{microtype}      % microtypography
\usepackage{xcolor}         % colors
\usepackage{multirow}
\usepackage{amsmath}
\usepackage[ruled]{algorithm2e} % Required for the algorithm environment
\usepackage{graphicx} % Required for inserting images
\usepackage{tcolorbox}
\usepackage{caption}
\usepackage{menukeys}
\usepackage{amsthm}
% \usepackage[margin=1in]{geometry} % Adjust margins to ensure the table fits well
\usepackage{tabularx} % Import tabularx for tables that can adjust their width automaticallyu
% \usepackage{wraptable}
\usepackage{wrapfig,lipsum,booktabs}
\usepackage{enumitem}

% \usepackage{algorithm}
% \usepackage{algpseudocode}

\usepackage{array}
\usepackage{amssymb}

\newcolumntype{M}[1]{>{\centering\arraybackslash}p{#1}}


\newcommand{\model}[0]{\textsc{AlphaLLM}}
\newcommand{\emcts}[0]{$\eta$\textsc{Mcts}}
\newcommand{\prm}[0]{\texttt{PRM}}
\newcommand{\orm}[0]{\texttt{ORM}}

\newcommand{\ie}[0]{\emph{i.e., }}
\newcommand{\ea}[0]{\emph{et al. }}
\newcommand{\eg}[0]{\emph{e.g., }}
\newcommand{\cf}[0]{\emph{cf. }}
\newcommand{\etc}[0]{\emph{etc.}}
\newcommand{\aka}[0]{\emph{a.k.a. }}
\newcommand{\RN}[1]{%
	\textup{\lowercase\expandafter{\it \romannumeral#1}}%
}
\newcommand{\enterkey}[0]{{\scriptsize{\keys{\return}}}}

\input{math_command}

% \title{Formatting Instructions For NeurIPS 2023}
\title{Toward Self-Improvement of LLMs via Imagination, Searching, and Criticizing}
% \title{A Preliminary exploration of LLMs Self-improvement with Search and Learning}

% The \author macro works with any number of authors. There are two commands
% used to separate the names and addresses of multiple authors: \And and \AND.
%
% Using \And between authors leaves it to LaTeX to determine where to break the
% lines. Using \AND forces a line break at that point. So, if LaTeX puts 3 of 4
% authors names on the first line, and the last on the second line, try using
% \AND instead of \And before the third author name.

% \author{Ye Tian\textsuperscript{1,2}\thanks{Equal Contribution; {\textdagger}Corresponding Author}, Baolin Peng\textsuperscript{1}\textsuperscript{$*$}, Linfeng Song\textsuperscript{1}\textsuperscript{$*$}, Lifeng Jin\textsuperscript{1}, Dian Yu\textsuperscript{1}, Lei Han\textsuperscript{2}, Haitao Mi\textsuperscript{1}\textsuperscript{\textdagger}, Dong Yu\textsuperscript{1}\\
% \textsuperscript{1}Tencent AI Lab, Bellevue, WA\\
% \textsuperscript{2}Tencent Robotics X \\
% \texttt{\{baolinpeng,lfsong,lifengjin,yudian,haitaomi,dyu\}@global.tencent.com} \\
% \texttt{\{yaptian,lxhan\}@tencent.com} \\\\
% }
\author{Ye Tian\textsuperscript{1,2}\thanks{Equal Contribution; {\textdagger}Corresponding Author}, Baolin Peng\textsuperscript{1}\footnotemark[1], Linfeng Song\textsuperscript{1}\footnotemark[1], Lifeng Jin\textsuperscript{1}, Dian Yu\textsuperscript{1}, Lei Han\textsuperscript{2}\\
\bf{Haitao Mi}\textsuperscript{1}\textsuperscript{\textdagger}, \bf{Dong Yu}\textsuperscript{1}\\
\textsuperscript{1}Tencent AI Lab, Bellevue, WA\\
\textsuperscript{2}Tencent Robotics X \\
\texttt{\{baolinpeng,lfsong,lifengjin,yudian,haitaomi,dyu\}@global.tencent.com} \\
\texttt{\{yaptian,lxhan\}@tencent.com} \\\\
}

% \author{%
%   David S.~Hippocampus\thanks{Use footnote for providing further information
%     about author (webpage, alternative address)---\emph{not} for acknowledging
%     funding agencies.} \\
%   Department of Computer Science\\
%   Cranberry-Lemon University\\
%   Pittsburgh, PA 15213 \\
%   \texttt{hippo@cs.cranberry-lemon.edu} \\
  % examples of more authors
  % \And
  % Coauthor \\
  % Affiliation \\
  % Address \\
  % \texttt{email} \\
  % \AND
  % Coauthor \\
  % Affiliation \\
  % Address \\
  % \texttt{email} \\
  % \And
  % Coauthor \\
  % Affiliation \\
  % Address \\
  % \texttt{email} \\
  % \And
  % Coauthor \\
  % Affiliation \\
  % Address \\
  % \texttt{email} \\
% }


\begin{document}
% \footnotetext{*Equal Contribution; \textsuperscript{\textdagger}Corresponding Author}

\maketitle


\begin{abstract}

% Despite the impressive capabilities of Large Language Models (LLMs) on various tasks, they still struggle with scenarios that involves complex reasoning and planning. Recent work proposed advanced prompting techniques and the necessity of fine-tuning with high-quality data to augment LLMs' reasoning abilities. However, these approaches are inherently constrained by data availability and quality. In light of this, self-correction and self-learning emerge as viable solutions, employing strategies that allow LLMs to refine their outputs and learn from self-assessed rewards. Yet, the efficacy of LLMs in self-refining its response, particularly in complex reasoning and planning task, remains dubious. In this paper, we introduce AlphaLLM, which integrates Monte Carlo Tree Search (MCTS) with LLMs to establish a self-improving feedback loop, thereby enhancing the capabilities of LLMs without additional annotations. Drawing inspiration from the success of AlphaGo, AlphaLLM addresses the unique challenges of combining MCTS with LLM for self-improvement, including data scarcity, the vastness search spaces of language tasks, and the subjective nature of feedback in language tasks. AlphaLLM is comprised of prompt synthesis module, an efficient MCTS approach tailored for language tasks, and a trio of critic models for precise feedback. Our experimental results in mathematical reasoning tasks demonstrate that AlphaLLM significantly enhances the performance of LLMs without additional annotations, shedding lights on the promise of self-improvement in LLMs. 

% Despite the impressive capabilities of Large Language Models (LLMs) on various tasks, they still struggle with scenarios that involves complex reasoning and planning. Recent work proposed advanced prompting techniques and the necessity of fine-tuning with high-quality data to augment LLMs' reasoning abilities. However, these approaches are inherently constrained by data availability and quality. In light of this, self-correction and self-learning emerge as viable solutions, employing strategies that allow LLMs to refine their outputs and learn from self-assessed rewards. Yet, the efficacy of LLMs in self-refining its response, particularly in complex reasoning and planning task, remains dubious. In this paper, we introduce \model{} for the self-improvements of LLMs, which integrates Monte Carlo Tree Search (MCTS) with LLMs to establish a self-improving loop, thereby enhancing the capabilities of LLMs without additional annotations. Drawing inspiration from the success of AlphaGo, \model{} addresses the unique challenges of combining MCTS with LLM for self-improvement, including data scarcity, the vastness search spaces of language tasks, and the subjective nature of feedback in language tasks. \model{} is comprised of prompt synthesis component, an efficient MCTS approach tailored for language tasks, and a trio of critic models for precise feedback. Our experimental results in mathematical reasoning tasks demonstrate that \model{} significantly enhances the performance of LLMs without additional annotations, showing the potential for self-improvement in LLMs.

Despite the impressive capabilities of Large Language Models (LLMs) on various tasks, they still struggle with scenarios that involves complex reasoning and planning. Self-correction and self-learning emerge as viable solutions, employing strategies that allow LLMs to refine their outputs and learn from self-assessed rewards. Yet, the efficacy of LLMs in self-refining its response, particularly in complex reasoning and planning task, remains dubious. In this paper, we introduce \model{} for the self-improvements of LLMs, which integrates Monte Carlo Tree Search (MCTS) with LLMs to establish a self-improving loop, thereby enhancing the capabilities of LLMs without additional annotations. Drawing inspiration from the success of AlphaGo, \model{} addresses the unique challenges of combining MCTS with LLM for self-improvement, including data scarcity, the vastness search spaces of language tasks, and the subjective nature of feedback in language tasks. \model{} is comprised of prompt synthesis component, an efficient MCTS approach tailored for language tasks, and a trio of critic models for precise feedback. Our experimental results in mathematical reasoning tasks demonstrate that \model{} significantly enhances the performance of LLMs without additional annotations, showing the potential for self-improvement in LLMs. The code is available at \url{https://github.com/YeTianJHU/AlphaLLM}.
  
\end{abstract}

\section{Introduction}
\label{sec:intro}
%LLMs, trained on trillions of tokens with billions of parameters have shown unparalleled capabilities in a wide range of natural language processing tasks~\cite{}. Nevertheless, significant attention have drawn from both the research community and industry on the concerns about their precision, complex reasoning, and planning capabilities ~\cite{}. While advanced prompting approaches such as Chain, Tree, Graph-of-Thought~\cite{}, which generate intermediate steps in the reasoning process demonstrate large improvements on reasoning capability of LLMs, it remains essential to fine-tune LLMs using a substantial volume of high-quality, supervised data to fundamentally improve the model performance~\cite{}. This methodology is inherently limited by the scope and quality of data that humans can provide~\cite{}.

% @article{huang2023large,
%   title={Large language models cannot self-correct reasoning yet},
%   author={Huang, Jie and Chen, Xinyun and Mishra, Swaroop and Zheng, Huaixiu Steven and Yu, Adams Wei and Song, Xinying and Zhou, Denny},
%   journal={arXiv preprint arXiv:2310.01798},
%   year={2023}
% }

% @inproceedings{clark2015training,
%   title={Training deep convolutional neural networks to play go},
%   author={Clark, Christopher and Storkey, Amos},
%   booktitle={International conference on machine learning},
%   pages={1766--1774},
%   year={2015},
%   organization={PMLR}

% @article{valmeekam2022large,
%   title={Large language models still can't plan (a benchmark for llms on planning and reasoning about change)},
%   author={Valmeekam, Karthik and Olmo, Alberto and Sreedharan, Sarath and Kambhampati, Subbarao},
%   journal={arXiv preprint arXiv:2206.10498},
%   year={2022}
% }
% @article{stechly2024self,
%   title={On the Self-Verification Limitations of Large Language Models on Reasoning and Planning Tasks},
%   author={Stechly, Kaya and Valmeekam, Karthik and Kambhampati, Subbarao},
%   journal={arXiv preprint arXiv:2402.08115},
%   year={2024}
% }

LLMs, trained on trillions of tokens with billions of parameters have shown unparalleled capabilities in a wide range of natural language processing tasks~\citep{touvron2023llama,team2023gemini,openai2023gpt}. Nevertheless, they continue to face challenges in scenarios requiring complex reasoning and strategic planning ~\citep{valmeekam2022large,stechly2024self}. While advanced prompting approaches such as Chain, Tree, Graph-of-Thought~\citep{wei2022chain,yao2024tree, besta2024graph, ding2023everything}, it remains essential to fine-tune LLMs using a substantial volume of high-quality, supervised data to fundamentally improve the model performance~\citep{nye2021show,lewkowycz2022solving,chung2022scaling}. This methodology is inherently limited by the scope and quality of data that humans can provide.

% LLMs, trained on trillions of tokens with billions of parameters have shown unparalleled capabilities in a wide range of natural language processing tasks~\citep{touvron2023llama,team2023gemini,openai2023gpt}. Nevertheless, they continue to face challenges in scenarios requiring complex reasoning and strategic planning ~\citep{valmeekam2022large,stechly2024self}. While advanced prompting approaches such as Chain, Tree, Graph-of-Thought~\citep{wei2022chain,yao2024tree, besta2024graph, ding2023everything}, which generate intermediate steps in the reasoning process demonstrate large improvements on reasoning capability of LLMs, it remains essential to fine-tune LLMs using a substantial volume of high-quality, supervised data to fundamentally improve the model performance~\citep{nye2021show,lewkowycz2022solving,chung2022scaling}. This methodology is inherently limited by the scope and quality of data that humans can provide.

Considering these challenges, the concept of self-correction and self-learning have been proposed as promising solutions~\citep{madaan2024self,saunders2022self,chen2024self}. Within these framework, LLMs typically operate by employing two main strategies: 1) they continuously refine their responses based on the feedback of their past responses, and 2) they extensively sample responses then learn from preferences judged by itself as reward models with PPO or DPO~\citep{yuan2024advancing,yuan2024self,chen2024self}. However, it remains a matter of ongoing research whether LLMs can effectively critique their own outputs to either enhance response quality or apply a scalar reward to indicate the quality of responses, especially in contexts demanding intricate planning and reasoning~\citep{valmeekam2022large,stechly2024self,huang2023large,hong2023closer}. On the other hand, advanced search algorithms such as MCTS, combined with reinforcement learning, have enabled models to learn from self-play and achieve human parity or even surpass human performance in complex tasks such as the game of Go~\citep{silver2016mastering, silver2017mastering}. This naturally raises a question: is it viable to leverage the strengths of MCTS alongside LLMs to inaugurate a novel paradigm of self-improving? More precisely, could the assimilation of MCTS empower LLMs to more effectively explore better responses, guided by strategic signals, and subsequently optimize these responses to enhance overall performance?

% To answer this question, we firstly take a systematic view of AlphaGo. There are three key elements lead to the success of it: 1) Large volume of expert and self-playing data; The imitation learning on expert data allows it to learn and mimic human-like moves, and the learning on self-play data encourages the discovery of innovative tactics beyond human limitations~\citep{clark2015training}. 2) MCTS that helps explore possible moves based on statistical sampling of the search space. By efficiently narrowing down the most promising moves and simulating their outcomes, AlphaGo could make highly informed decisions even in the complex and vast decision space. 3) Accurate and distinct environment feedback; The clear and accurate feedback (win or loss) from the environment in Go provided AlphaGo with a straightforward and unambiguous signal to learn from. Taking these key elements into considerations, the integration of MCTS for LLM self-improving introduces several challenges: 1) Limited Data: The high-quality annotation data for LLMs is typically limited.  In addition, the synthesis of effective training data for LLMs, akin to AlphaGo's self-play, remains a research question. 2) Search efficiency: The potential token combinations in natural language tasks create an exponentially large search space, presenting a formidable obstacle to the efficiency of MCTS. 3) Imperfect feedback: Unlike the clear win/loss signal in Go, feedback in natural language tasks is often subjective and nuanced, lacking a straightforward metric for success. 

To answer this question, we begin with a systematic examination of AlphaGo, identifying three critical aspects for its success: (\RN{1}) The large volume of data, including self-play data. (\RN{2}) The use of tree search, which facilitates the exploration of potential moves through statistical sampling of the large search space. (\RN{3}) Accurate and unambiguous environment feedback; the direct and accurate feedback (win or loss) provided by the game of Go offers a clear and unequivocal learning signal~\citep{silver2017mastering}. The integration of MCTS with LLMs for self-improvement has several challenges: (\RN{1}) Limited Data: High-quality annotated data for LLMs is generally scarce. Furthermore, how to construct of synthetic data for LLMs training, similar to AlphaGo's self-play data, remains unclear. (\RN{2}) Search Efficiency: The vast number of potential token combinations in natural language tasks results in an exponentially large search space, posing a significant challenge to the efficiency of MCTS~\citep{Ramamurthy2022IsRL}. (\RN{3}) Imperfect Feedback: In contrast to the clear win/loss feedback in Go, feedback in natural language tasks is often subjective and nuanced, without a straightforward measure of success.

% To answer this question, we begin with a systematic examination of AlphaGo, identifying three critical aspects for its success: (\RN{1}) The large volume of expert and self-play data; imitation learning on expert data enables it to simulate human-like strategies, and the reinforcement learning on self-play data fosters the emergence of novel tactics that surpass human capabilities~\citep{clark2015training}. (\RN{2}) The use of tree search, which facilitates the exploration of potential moves through statistical sampling of the large search space. This approach allows AlphaGo to effectively identify and simulate the most promising strategies, thereby making highly informed decisions in the complex and vast decision space~\citep{silver2016mastering}. (\RN{3}) Accurate and unambiguous environment feedback; the direct and accurate feedback (win or loss) provided by the game of Go offers a clear and unequivocal learning signal~\citep{silver2017mastering}. The integration of MCTS with LLMs for self-improvement has several challenges: (\RN{1}) Limited Data: High-quality annotated data for LLMs is generally scarce. Furthermore, how to construct of synthetic data for LLMs training, similar to AlphaGo's self-play data, remains unclear. (\RN{2}) Search Efficiency: The vast number of potential token combinations in natural language tasks results in an exponentially large search space, posing a significant challenge to the efficiency of MCTS~\citep{Ramamurthy2022IsRL}. (\RN{3}) Imperfect Feedback: In contrast to the clear win/loss feedback in Go, feedback in natural language tasks is often subjective and nuanced, without a straightforward measure of success.

\begin{figure}[!t]
    \centering
    \includegraphics[width=0.9\textwidth]{figures/framework_crop.pdf}
    \caption{Imagination-Searching-Criticizing self-improvement loop: Imagination component synthesizes prompts as new learning examples, with MCTS searching better trajectories guided by signals from critics for policy improving.}
    \label{fig:framework}
\end{figure}

In this paper, we introduce \model{}, an imagination-searching-criticizing framework designed for the self-improvement of LLMs . \model{} consists of three key components, as illustrated in Figure~\ref{fig:framework}. First, an imagination component is designed to synthesize prompts, alleviating the issues of data scarcity. Second, we propose \emcts{} tailored for efficient searching in language tasks. Particularly, it has been show that planning at multiple levels of temporal abstraction is critical for RL problems with a long horizon and large action space~\citep{sutton1999between,peng2017composite,Luketina2019ASO}. As such, we propose formulating the text generation process as options over a Markov Decision Process (MDP) problem, where each option represents the generation of a collection of tokens for a specific subtask, similar to the concept of chains in chain-of-thought prompting. This formulation improves search efficiency by substantially reducing the search depth. Additionally, we propose the use of state merge and adaptive branching factors to further enhance search efficiency by balancing the trade-off between search width and depth. Lastly, since accurate feedback is crucial to the success of MCTS, we introduce a trio of critic models to guide \emcts{}, including a value function for estimating expected rewards, a process reward model for assessing node correctness, and an outcome reward model for evaluating the overall trajectory. For complex tasks with which LLMs struggle assessing such as arithmetic computation and code execution, to ensure the accuracy of feedback, we augment the critics with the capacity to make dynamic decisions on which tools to use, when to use them, and how to use them effectively. After \emcts{} stage, we collect the trajectory with the largest reward from the critic models as the training examples to improve LLMs. 

The experimental results on mathematical reasoning tasks demonstrate that \model{} can efficiently search for better responses and use them to improve LLMs' performance, forming an effective self-improving loop. Notably, based on Llama-2-70b and WizardMath-70B-V1.0, \model{} can improve its performance from 57.8 to 92.0 on GSM8K and from 20.7 to 51.0 on MATH, performing comparably to GPT-4. 

% In summary, our contributions are threefold:
% \begin{itemize}
% \item We examine the inherent challenges in harnessing AlphaGo's self-learning algorithms for LLMs, which are data scarcity, the complexity of search spaces, and the nuanced nature of feedback.
% \item We introduce \model{}, an imagination-searching-criticizing framework that integrates MCTS with LLMs, enabling them to self-improve without the need for additional annotations
% \item Experiments on mathematical reasoning problems show that, by employing \model{}, we can significantly enhance the performance of LLaMA-2 70B, elevating it to levels comparable with GPT-4 on the GSM8K and MATH datasets when \emcts{} decoding is utilized.
% \end{itemize}
% \begin{itemize}\setlength{\itemsep}{0pt}
% \item We analyze the challenges of applying AlphaGo style self-learning algorithm to LLMs and introduce \model{}, a novel framework that synergizes MCTS with LLMs for self-improvement.
% \item We integrate
% \item Experimental results on mathematical reasoning showcases remarkable performance gains of using MCTS for self-improving, validating the efficacy of \model{}.
% \end{itemize}


% propose to improve MCTS efficiency by formulating the task as options over MDPs problems. This formulation improves search efficiency by utilizing options, which are temporal abstractions of multiple tokens, rather than a single token as the search nodes. Additionally, techniques such as state fusion and adaptive branching factors are implemented to improve search efficiency by balancing the trade-off between search breadth and depth. 


% \begin{itemize}
%     \item LLM
%     \item LLM struggles at planning (left-to-right next token prediction)
%     \item AlphaX + LLM has several challenges
%     \item AlphaX: Action is clearly defined; Signal is 100\% accurate;
%     \item AlphaX + LLM:
%     \item Search
%     \begin{itemize}
%         \item Step: Options over MDP
%         \item Signal: Specialized statistical model + Symbolic engines (fast-rollout)
%         \item Efficient MCTS: state merger, adaptive exploration
%         \item slow/fast thinking ()
%     \end{itemize}
%     \item Learning
% \end{itemize}

\section{Related Work}
\label{sec:related_work}
\section{Related Work}\label{app:sec:related}

\paragraph{Generalist agents.} The AI community has witnessed the rising generalist models in both vision~\citep{lu2023unified,wang2023images,kirillov2023segment} and language~\citep{openai2022chatgpt,openai2023gpt4} domains. A generalist agent requires additional embodiment knowledge to interact with the environment and complete embodied acting tasks. Existing efforts towards generalist agents include: grounded reasoning and task planning in the real world~\citep{ahn2022can,huang2022inner}, skill generalization in open-world environment~\citep{fan2022minedojo,cai2023open,wang2023describe,wang2023voyager,cai2023groot,jxma_llm_vla_vlm_mas_multiagent_2023}, general robotic manipulation~\citep{brohan2022rt,jiang2023vima,gong2023arnold}, and unified vision-language-action (VLA) models such as Gato~\citep{reed2022generalist}, PaLM-E~\citep{driess2023palm}, EmbodiedGPT~\citep{mu2023embodiedgpt}, and RT-2~\citep{brohan2023rt}. \agent belongs to the \ac{vla} model, however, its goal is to build a generalist agent that can understand the real 3D world beyond 2D images, which is absent in existing works.

\paragraph{Multi-modal instruction tuning.} Pre-trained LLMs demonstrated practical for solving vision-language tasks~\citep{tsimpoukelli2021multimodal,alayrac2022flamingo,guo2023images,li2023blip,jxma_vlm_multimodal_2023}. Meanwhile, the instruction-tuning paradigm exhibited strong zero-shot generalization in NLP tasks~\citep{wei2022finetuned,sanh2022multitask,ouyang2022training,chung2022scaling}. The two streams merged into instruction-tuned LVLMs~\citep{liu2023visual,zhu2023minigpt,ye2023mplug,gao2023llama,li2023otter,gong2023multimodal,dai2023instructblip}. Despite the burst, these models are confined to 2D visual modalities, \eg, image or video. Concurrent works~\citep{yin2023lamm,hong20233d,wang2023chat,xu2023pointllm} extend to 3D vision tasks, but these models either lack the acting capability or unified efficient architecture.

\paragraph{Grounded 3D scene understanding.}
One key obstacle to building \agent is grounding the 3D world with natural languages. There exist diverse methods of grounded scene understanding, \eg, spatial relation modeling \citep{zhao20213dvg,chen2022language,zhu20233d} and fine-grained open-scene understanding \citep{peng2023openscene,kerr2023lerf}. However, due to data scarcity, how to utilize \ac{llm} to ground the 3D scene is rarely explored. Recently, 3D-LLM \citep{hong20233d} leverages multi-view images and Chat-3D~\citep{wang2023chat} uses object-centric point clouds to enable the \ac{llm} with 3D grounding. In this work, we devise both 2D and 3D encoders for grounding various visual representations and employ LoRA~\citep{hu2022lora} to efficiently fine-tune the \ac{llm}.

\paragraph{3D data prompting from LLMs.} LLMs exhibit extraordinary capabilities of text generation and serve as a source for collecting diverse instruction-following data~\citep{wang2023self,alpaca,peng2023instruction}. However, the lack of access to visual modalities makes it troublesome to collect visual instruction-tuning data. To address this issue, existing methods provide bounding boxes~\citep{liu2023visual} and add dense captions~\citep{li2023mimic,liu2023aligning} as image descriptions or directly use off-the-shelf \ac{lvlm}~\citep{zhu2023chatgpt,luo2023scalable} to help collect such data. Unlike concurrent attempts~\citep{yin2023lamm,hong20233d,wang2023chat} in collecting 3D instruction-tuning data, our approach features a scene-graph-based prompting and refinement method to prompt and correct the data.


\section{Preliminaries}
\label{sec:pre}
\subsection{Problem Formulation}

In this paper, we consider a LLM characterized by probability $p_\theta$ and denoted as policy $\pi_\theta$. It takes a sequence $\vx =[x_1, \cdots, x_n]$ as input, which is typically referred as prompt, to generate the response $\vy = [y_1, \cdots, y_m]$. In the context of LLMs, each $x_i$ and $y_i$ represents a token from a pre-defined vocabulary. The policy $\pi_\theta$ operates in an autoregressive manner, where each token is generated sequentially, relying solely on the context provided by the previously generated tokens. The policy therefore constitutes a Markov process in which the conditional probability distribution $p_\theta(\vy|\vx)$ can be decomposed and expressed with the chain rule as $p_\theta(\vy|\vx) = \prod_{i=1}^{m} p_{\theta}(y_i|\vx, \vy_{<i})$.
% \begin{equation*}
% p_\theta(\vy|\vx) = \prod_{i=1}^{m} p_{\theta}(y_i|\vx, \vy_{<i})
% \end{equation*}

With this property, the text generation task can be formulated as an Markov Decision Process (MDP) problem consisting of $(\gS, \gA, T, R, \gamma)$~\cite{} in which, $\vs_t \in \gS$ represents the context information of current trajectory, \ie current status of the generation process, \eg a partial response to a prompt; $a_t \in \gA$ denotes a single action or sampled token from the vocabulary, leading to a transition to a new state $\vs_{t+1}$, by concatenating $\vs_t$ and $a_t$; $r_t = R(\vs_t, a_t)$ manifest the evaluation of the generation to the prompt, reflecting the desirability or preferences of each state-action pair.


% \begin{itemize}
% \item {\bf State} $\vs_t \in \gS$: Represents the context information of current trajectory, \ie current status of the generation process, \eg a partial response to a prompt. The initial state \(s_0\) corresponds to the original prompt.
% \item {\bf Action} $a_t \in \gA$: Denotes a single action or sampled token from the vocabulary, leading to a transition to a new state $\vs_{t+1}$, by concatenating $\vs_t$ and $a_t$.
% \item {\bf Reward} $r_t = R(\vs_t, a_t)$: Manifest the evaluation of the generation to the prompt, reflecting the desirability or preferences of each state-action pair, such as whether the actions follow instructions in the prompt. 
% \end{itemize}
% \noindent $\gamma$ denotes the discount factor, while $T$ here signifies the transition probability function. We omit its detailed description as in tex

% \begin{itemize}
% \item {\bf State} $\vs_t \in \gS$: Represents the context information of current trajectory, \ie current status of the generation process, \eg a partial response to a prompt. The initial state \(s_0\) corresponds to the original prompt.
% \item {\bf Action} $a_t \in \gA$: Denotes a single action or sampled token from the vocabulary, leading to a transition to a new state $\vs_{t+1}$, by concatenating $\vs_t$ and $a_t$.
% \item {\bf Reward} $r_t = R(\vs_t, a_t)$: Manifest the evaluation of the generation to the prompt, reflecting the desirability or preferences of each state-action pair, such as whether the actions follow instructions in the prompt. 
% \end{itemize}
% \noindent $\gamma$ denotes the discount factor, while $T$ here signifies the transition probability function. We omit its detailed description as in text generation environment the transition is deterministic. 

% This MDP framework sets the stage for applying Reinforcement Learning (RL) methods to optimize the policy $\pi_\vtheta$ aiming to maximize the expected cumulative reward $R$. Within these setups, we describe the self-improving process of \model{} as follows: Starting with an initial dataset $\gD^0 = \{(\vx_i^0, \vy_i^0) \mid i \in [N]\}$ comprising N expert-generated prompt-response pairs, we first train critic models as reward $R$ manifesting the task success metrics. Subsequently, {\tt Synthesizer} constructs synthetic prompts $[\vx^1_i]$ as the novel learning materials. We then collect trajectories $[\hat{\vy}^1_i]$ via \emcts{} that are evaluated to have the highest reward by sampling policy $\pi^0_\theta$, forming a dataset $\gD^1 = \{(\vx_i^1, \hat{\vy}_i^1) \mid i \in [M]\}$. Finally, the policy $\pi^0_\theta$ is updated to maximize the expected reward on $\gD^1$. This iterative improvement process, detailed in \Algref{algo:self_improving}, aims to incrementally maximize the task-specific reward by synthesizing suitable prompts, employing efficient search algorithm, and utilizing precise evaluation by the critic models. We describe in details about the data synthesizer, efficient MCTS and the design of critic models in the subsequent sections.
%The process can be iterated multiple rounds, as described in \Algref{algo:self_improving}. Our aim is to

This MDP framework sets the stage for applying Reinforcement Learning (RL) methods to optimize the policy $\pi_\vtheta$ aiming to maximize the expected cumulative reward $R$. Base on these setups, we describe the self-improving problem. Given a LLM $\pi_\vtheta$ and an initial dataset $\gD^0$, which consists of $N$ expert-generated prompt-response pairs $\{(\vx_i^0, \vy_i^0) \mid i \in [N]\}$, the goal of self-improving is to iteratively refine $\pi_\theta$ to maximize the reward. The refinement process includes learning from synthesized prompts and corresponding responses. These responses are obtained using an advanced search algorithm that navigates the space of possible responses to maximize the expected reward. The detailed process is described in \Algref{algo:self_improving} in Appendix. The primary challenges in forming an effective self-improving loop lie in synthesizing suitable prompts, efficiently searching over a vast action space, and obtaining precise feedback, which will be discussed in \S \ref{sec:method}.

% how to synthesize appropriate prompts, efficiently search over large action space, and obtain accurate feedback.

% process of \model{} as follows: Starting with an initial dataset $\gD^0 = \{(\vx_i^0, \vy_i^0) \mid i \in [N]\}$ comprising N expert-generated prompt-response pairs, we first train critic models as reward $R$ manifesting the task success metrics. Subsequently, {\tt Synthesizer} constructs synthetic prompts $[\vx^1_i]$ as the novel learning materials. We then collect trajectories $[\hat{\vy}^1_i]$ via \emcts{} that are evaluated to have the highest reward by sampling policy $\pi^0_\theta$, forming a dataset $\gD^1 = \{(\vx_i^1, \hat{\vy}_i^1) \mid i \in [M]\}$. Finally, the policy $\pi^0_\theta$ is updated to maximize the expected reward on $\gD^1$. This iterative improvement process, detailed in \Algref{algo:self_improving}, aims to incrementally maximize the task-specific reward by synthesizing suitable prompts, employing efficient search algorithm, and utilizing precise evaluation by the critic models. We describe in details about the data synthesizer, efficient MCTS and the design of critic models in the subsequent sections.



\subsection{Monte Carlo Tree Search}

MCTS is a sampling-based search algorithm for policy optimization in decision-making problems. It would iteratively build a search tree, by repeating four phases: selection, expansion, evaluation, and backpropagation. In the selection phase, it would recursively select the children from the root node by Upper Confidence Bound (UCB) ~\citep{auer2002finite}, $UCB(i)=w_i+C*\sqrt{2*\ln{\frac{N_i}{n_i}}}$, where $n_i$ and $N_i$ are the visit counts for the node $i$ and its parent respectively, $C$ represents a hyperparameter balancing exploration and exploitation, and the $w_i$ is the average value of all descendant nodes of $i$.
% \begin{equation}
% \label{eqs:ucb}
% UCB(i)=w_i+C*\sqrt{2*\ln{\frac{N_i}{n_i}}}
% \end{equation}
% where $n_i$ and $N_i$ are the visit counts for the node $i$ and its parent respectively, $C$ represents a hyperparameter balancing exploration and exploitation, and the $w_i$ is the average value of all descendant nodes of $i$. %Following selection, the tree undergoes expansion according to the defined policy in the expansion phase. Then in the evaluation phase, the value of the newly expanded node is estimated, by sampling or model-based methods. Finally, in the backpropagation phase, the estimated value is backpropagated to all ancestor nodes of the newly expanded node. 

\section{\model{}}
\label{sec:method}


% Algorithm 1 - Self improving loop

% Require : Initial dataset D^0 = {(x_i, y_i) | i \in [N]}, policy model \pi_\theta^0, reward model R, number of self-improving training loop K

% for k ← 1, …, K  do

% Generate synthetic prompts [x_i^1] = Data(\pi_\theta^0, D^0)

% Collect trajectories with MCTS guided by R [y_i^1] = MCTS(\pi_\theta^{k-1}, [x_i^1])

% Construct dataset D^k = {(x_i^k, y_i^k) | i \in [N]}

% Update policy \theta^k = argmin_\theta L(\pi_\theta^k-1, D^k)

% end for

% output \theta^k


% - self-improving problem formulation, transiting to data / search / critic
% - overview
% - data
% - efficient mcts
% -- formulation, s,o,a,r,beta t
% -- mcts
% -- adaptive branching
% -- state fuse
% -- fastrollout
% - Critic Models
% - - value function
% - - process rm
% - - outcome rm

\subsection{Overview}

The architecture of \model{} is depicted in Figure~\ref{fig:framework}, comprising three key components. Firstly, the imagination component is tasked with synthesizing prompts as learning examples. Secondly, an efficient search component, named \emcts{}, is proposed to search high-quality trajectories for optimizing the policy. Lastly, the search process is guided by critics specifically designed to provide reliable signals.

\subsection{Data Synthesizing}

Let $\gD^0 = \{(\vx_i, \vy_i) \mid i \in [N]\}$ denote the initial dataset consisting of $N$ expert-generated prompt-response pairs. The data synthesizing process aims to expand this dataset by generating a set of synthesized prompts $\gD^1 = \{(\vx_i^1,\cdots) \mid i \in [N]\}$. The generation of each synthesized prompt $\vx_i^1$ can be mathematically described as a transformation $g$ applied to one or more examples from $\gD^0$, $\vx_i^1 = g(\vx_{i_1}^0,\cdots,\vx_{i_m}^0, \pi^0)$
% \begin{equation*}
% \vx_i^1 = g(\vx_{i_1}^0,\cdots,\vx_{i_m}^0, \pi^0)
% \end{equation*}
where $\vx_{i_1}^0,\cdots,\vx_{i_m}^0$ are selected examples from $\gD^0$. The transformation function $g$ controls the synthesis process, which can be a learnable function, manually defined heuristic rules, a strong LLM or the policy model itself $\pi^0$ equipped with data synthesis instructions. The data synthesizing process aims to enrich the diversity and complexity presented for the training of the policy model. Among various strategies, such as Self-instruct~\citep{wang2022self}, Evol-instruct~\citep{xu2023wizardlm}, we opt for a method akin to that described in~\cite{yu2023metamath}.

% Challenges and solutions
% 1) Naive MCTS at token level for LLM tasks
% MCTS is a sampling-based search algorithm for policy optimization in decision-making problems. It would iteratively build a search tree, by repeating four phases: selection, expansion, evaluation, and backpropagation. In the selection phase, it would recursively select the children from the root node by Upper Confidence Bound (UCB) bandit ~\cite{auer2002finite}, which is 
% \begin{equation*}
% UCB(i)=w_i+C*\sqrt{2*\ln{\frac{N_i}{n_i}}}
% \end{equation*}
% where $n_i$ and $N_i$ are the visit counts for the node $i$ and its parent respectively, $C$ represents a hyperparameter, and the $w_i$ is the average value of all descendant nodes of $i$. Following selection, the tree undergoes expansion according to the defined policy in the expansion phase. Then in the evaluation phase, the value of the newly expanded node is estimated, by sampling or model-based methods. Finally, in the backpropagation phase, the estimated value is backpropagated to all ancestor nodes of the newly expanded node. 

% To apply MCTS with large language models (LLMs), it is a natural approach ~\cite{liu2023making} to perform search at the token-level, considering each token as an action. However, due to the typically large vocabulary size in LLMs, conducting a deep search becomes very hard as the search space grows exponentially.

% Solution: options
% Following~\cite{option_mcts, de2016monte}, we adapt the term options to represent courses of actions. Typically, an option $o = \langle \gI, \pi, \beta \rangle$, where $\gI \subseteq \gS$ is a set of initial states for the option; $\pi: \gS \times \gA \rightarrow [0,1]$ is a policy to generate actions, which in our case is a LLM; and $\beta: \gS^{+} \rightarrow [0,1]$ is the termination function. Starting from a state $s_t$, we can choose all the options for which $s_t \in \gI$. Once an option is chosen, the policy $\pi$ will generate actions for several steps until the option terminates according to the termination function $\beta$.

% We can then construct the MCTS at the option-level. This means that a new node expansion requires the generation of a complete option, rather than a single token. As a result, the search space is significantly reduced, since the number of options in a trajectory is much fewer than the number of tokens. This allows for a deeper search, more coverage of the search space, and reduces the frequency to request the feedback models, such as the value model. 

% % Policy over options: $\mu: S \times O \rightarrow [0, 1]$ %

% In previous works such as AlphaZero~\cite{silver2018general}, a search tree is built at every step, and the next action is selected based on certain root node statistics, such as visitation frequency. After performing the selected action and receiving the updated state from environment, the tree is built again. This approach is adopted due to the stochasticity of the environment, which requires building the tree based on the most recent actual move. However, in the case of text generation, as the state transition $T$ is deterministic, it is possible to build the tree only once, since all states are accurate. Starting from the initial state, the select-expand-evaluate-backup process is performed several times until a specified condition is met, such as the maximum number of rollouts or terminated nodes. The final answer is then selected from the tree by choosing the optimal trajectory. This offline version of MCTS has also been employed in~\cite{feng2023alphazero}.

% Qs: 
% 2) high efficient MCTS for LLMs
% 3) feedback single for MCTS
% 4) learning

% \subsection{Option-level MCTS}
% % problem defination
% The math problem can be formulated as a Markov Decision Process (MDP): $(S,A,T,R,\gamma)$. The state space $S$ contains information about the math question, as well as answers from previous steps. The action, in the case of segment-level MCTS, is a step of output, thus will be lie in a very large action space $A$. $T$ represent the transition function, which in this case is deterministic. We will only receive reward in the end of each trajectory from reward function $R$, and with the discount factor $\gamma=1$. For each step, we aim to find an optimal action of output that maximize the expected reward $E_{a\sim \pi(a|s)}[\sum_{i=0}^{\infty}\gamma^{i}R(s_{t+1},a_{t+i})]$, where $\pi(a|s)$ is our current policy to solve the problem.

% use Option to defined V/Q


% \subsection{Efficient MCTS}
\subsection{\emcts{}}
\label{sec:mcts}

% \begin{figure}[!t]
    \centering
    \includegraphics[width=\textwidth]{figures/emcts.pdf}
    \caption{An overview of the four operations of \emcts{}. A node is selected, expanded, simulated with fast rollout policy until a terminal node is reached, then the signals from value function, \prm{} and \orm{} are backpropagated.}
    \label{fig:emcts}
\end{figure}

\subsubsection{Option-level MCTS}

\begin{table}[!htb]
\footnotesize
    \centering
    \setlength{\tabcolsep}{4pt}
    \begin{tabular}{c|c|c}

    \toprule
    \texttt{Search Node} & \texttt{Example} & \texttt{Termination}  \cr
    \midrule
    Token-level & $y_0 \rightarrow y_1 \rightarrow y_2 \rightarrow y_3 \rightarrow y_5 \rightarrow y_6 \rightarrow y_7 \rightarrow y_8$ &  token\cr
    \midrule
    Sentence-level & $y_0 y_1 y_2$ \enterkey{}  $\rightarrow y_4 y_5 y_6$ \enterkey{} $\rightarrow y_7 y_8 y_9 y_{10}$ & new line\cr
    \midrule
    Option-level & $y_0$  $\rightarrow y_1 y_2$ \enterkey{} $\rightarrow y_4 y_5 y_6$ \enterkey{} $y_7 y_8 y_9$ \enterkey{} $\rightarrow y_{10}$& termination function\cr
    \bottomrule
    \end{tabular}
    \vspace{2mm}
    \caption{Comparative illustration of token-level, sentence-level, and option-level MCTS search nodes. $y$ denotes a token sampled from the policy model. The arrow $\rightarrow$ represents the transition from one search node to the subsequent node within the search process.}
    \label{tab:option}
\end{table}


% To apply MCTS to LLMs, it is naturally ~\cite{liu2023making} to performing search at the token-level, considering each token as an action. However, due to the typically large vocabulary size in LLMs, conducting a deep search becomes very challenging as the search space grows exponentially. Some attempts propose sentence-level search where each sentence or step is regarded as an action. While sentence-level search can reduce the complexity of the search space, it might affect the flexibility and performance applying MCTS to LLM as for some tasks subtle variations of several tokens could significantly alter the reward and some tasks might requires more than sentence level search, such as paragraph level search.

% However, the typically large vocabulary of LLMs poses a significant challenge, as conducting a deep search becomes increasingly difficult due to the exponential growth of the search space.

When applying MCTS to LLMs, it is natural to perform token-level search, where each token is considered as an action~\citep{liu2023making}. However, the substantial vocabulary size typical of LLMs presents a significant challenge \ie conducting a deep search in such a vast space becomes increasingly complex as the search space expands exponentially. To mitigate this, some efforts proposed a sentence-level search, treating each sentence or step as a search node~\citep{feng2023alphazero}. While this method reduces the search space, it might compromise the flexibility and effectiveness of applying MCTS to LLMs, which is particularly true for tasks where subtle variations in token can dramatically impact the outcome, or where a more comprehensive search beyond a sentence is necessary.

Inspired by~\cite{option_mcts, de2016monte}, we use the term option as a search node and propose option-level MCTS where each option represents a sequence of tokens, which can range from multiple tokens to several sentences. A comparisons of different levels search is listed in Table~\ref{tab:option}. Mathematically, an option $o = \langle \gI, \pi, \beta \rangle$, where $\gI \subseteq \gS$ is a set of initial states for the option; $\pi: \gS \times \gA \rightarrow [0,1]$ is a policy to generate actions, which in our case is a LLM; and $\beta: \gS^{+} \rightarrow [0,1]$ is the termination function. Starting from a state $s_t$, we can choose all the options for which $s_t \in \gI$. Once an option is chosen, the policy $\pi$ will generate actions for several steps until the option terminates according to the termination function $\beta$. The option-level MCTS consists of stages including selection, expansion, simulation, and backpropagation. The option-level formulation offers more flexibility compared to the sentence-level, as a new line can be treated as a special case of the termination function, as demonstrated in Table \ref{tab:option}. Additional detailed steps of the option-level MCTS can be found in Appendix \ref{app:option_level_mcts}.

% As illustrated in Figure~\ref{fig:emcts}, option-level MCTS consists of the following operations:
% \begin{itemize}[noitemsep,topsep=0pt,parsep=2pt,partopsep=0pt,leftmargin=*]
% \item \textbf{Selection} Starting from the root node, we iteratively select the child node based on Equation \ref{eqs:ucb}.
% \item \textbf{Expansion} Once an expandable leaf node is selected, a new node is generated by starting with the previous state of the parent node as the initial option state. The option is then sampled using the policy $\pi$, and its completion is determined by the termination function $\beta$. 
% \item \textbf{Simulation} The scaled reward of the newly expanded node, as well as some simulated future trajectories are evaluated using the feedback functions, which will be discussed in \S \ref{sec:critic}.
% \item \textbf{Backpropagation} The average value of the newly generated node and all its ancestors is updated using the scaled reward from the evaluation step. Meanwhile, the visit counts for these nodes are also increased by one.
% \end{itemize}

% \begin{itemize}[noitemsep,topsep=0pt,parsep=0pt,partopsep=0pt]
%   \item \paragraph{Selection} Starting from the root node, we iteratively select the child node based on Equation \ref{eqs:ucb}.
% \end{itemize}

%   \item \paragraph{Selection} Starting from the root node, we iteratively select the child node based on Equation \ref{eqs:ucb}.

% \item \paragraph{Expansion} Once an expandable leaf node is selected, a new node is generated by starting with the previous state of the parent node as the initial option state. The option is then sampled using the policy $\pi$, and its completion is determined by the termination function $\beta$. 
%   \item \paragraph{Simulation} The scaled reward of the newly expanded node, as well as some simulated future trajectories are evaluated using the feedback functions, which will be discussed in \S \ref{sec:critic}.
%   \item \paragraph{Backpropagation} The average value of the newly generated node and all its ancestors is updated using the scaled reward from the evaluation step. Meanwhile, the visit counts for these nodes are also increased by one.

% Employing an option to substitute a single token within each node could reduces search space, as the number of options in a trajectory is much smaller than the number of tokens. This facilitates a deeper search, broader coverage of the search space, and minimizes the frequency of requesting feedback from functions such as the value model. Moreover, the option-level offers more flexibility compared to the sentence-level, as a new line can be treated as a special case of the termination function, as demonstrated in Table \ref{tab:option}.

% Policy over options: $\mu: S \times O \rightarrow [0, 1]$ %

% --
% In previous works such as AlphaZero~\cite{silver2018general}, a search tree is built at every step, and the next action is selected based on certain root node statistics, such as visitation frequency. After performing the selected action and receiving the updated state from environment, the tree is built again. This approach is adopted due to the stochasticity of the environment, which requires building the tree based on the most recent actual move. However, in the case of text generation, as the state transition $T$ is deterministic, it is possible to build the tree only once, since all states are accurate. Starting from the initial state, the select-expand-evaluate-backup process is performed several times until a specified condition is met, such as the maximum number of rollouts or terminated nodes. The final answer is then selected from the tree by choosing the optimal trajectory. This offline version of MCTS has also been employed in~\cite{feng2023alphazero}.


% \subsubsection{Importance Weighted Expansion}
% % % Importance weighted expansion
% In previous work related to option/sentence level tree search ~\cite{feng2023alphazero, yao2024tree}, it has been a common practice to assume that each node in the tree has the same predefined width \ie branching factor. This is due to the fact that unlike token-level MCTS with a limited action space, the sample space at the option-level is exceedingly large, with an unlimited number of token combinations. Consequently, it is necessary to set a predefined maximum width and often result in an inefficient search space. For a given node, the error lead by the the branching factor limit can be formulated as
% \begin{equation*}
% E_{\phi} = \mathop{\mathbb{E}}[\min_{i}|v_{\phi}^{\pi}([\vs_t,\vo_t^{new}])-v_{\phi}^{\pi}([\vs_t,\vo_t^{i}])|]
% \end{equation*}
% where $\vo_t^{new}$ is the option of a potential new children that can not be added, $\vo_t^{i}$ represents the current ith children, and $v_{\phi}^{\pi}$ is the value function which will be detailed in \S \ref{sec:critic}. For now we assume that the value of the maximum $m_t$ children nodes are uniform distribution (which is not true but we will cover this in the next part). It is straightforward to show that 
% \begin{equation*}
% E_{\phi} \le \frac{\max_{\vo_t} |v^{\pi}([\vs_t,\vo_t])- v^{\pi}(\vs_t)|}{m_t}.
% \end{equation*}
% While the $E_{\phi}$ is less than some $\epsilon$, we want to use less total number of node for efficiency. And this thus can be formulate as an optimization problem:
% \begin{align*}
% \text{minimize:} & \sum m_t\\
% \text{subject to:} & \frac{\max_{i} |v^{\pi}([\vs_t,\vo_t^{i}])- v^{\pi}(\vs_t)|}{m_t} \le \epsilon
% \end{align*}
% We showed in Appendix that the optimal choice of $m_t$ is proportional to $\frac{1}{m_t}\max_{i} |v^{\pi}([\vs_t,\vo_t^{i}])- v^{\pi}(\vs_t)|$ constant of $1/(\epsilon)$. We name this part as importance $I(\vs_t)$, of the node $\vs_t$.
% \begin{equation*}
% % I(s_t) = \max_{a_t} |Q^{\pi}(s_t,a_t)- E_{a\sim\pi}[Q^{\pi}(s_t,a_t)]|
% I(\vs_t) = \max_{\vo_t} |v^{\pi}([\vs_t,\vo_t])- v^{\pi}(\vs_t)|
% \end{equation*}

% % A more effective and efficient way to determine the branching factor for each node is to dynamically adjust it based on the importance of each node. This approach allows us to allocate a larger child budget to nodes of higher importance, thereby preventing inefficient exploration of these nodes and ensuring that we do not miss promising solutions. Meanwhile, by reducing the number of children for less important nodes, we can perform deeper searches at various levels of the tree, rather than considering all possible options at each node. 

% Similar concept has also be proposed in ~\cite{taylor2014reinforcement, clouse1996integrating}. Intuitively, $I(\vs_t)$ captures the maximum value deviation from the current state. When this value is small, there is no need to explore further on this node, as there will not be a significant difference by rolling out on this node. Conversely, if the value is large, it is worth trying different children. We set the number of children allowed for a node $n(\vs_t)$ to be linear with this importance, using a factor $\alpha$. In practice, to avoid extreme cases of large variance of $I(\vs_t)$ in the early stage, we bound the number of children by depth-dependent constants $c_{\mathtt{min}}(t)$ and $c_{\mathtt{max}}(t)$:
% \begin{equation*}
% n(\vs_t) = \max\left(c_{\mathtt{min}}(t), \min\left(\lfloor \alpha I(\vs_t) \rfloor, c_{\mathtt{max}}(t)\right)\right).
% \end{equation*}

% \subsubsection{Efficient MCTS}
\subsubsection{Importance-Based Adaptive Branching}

In previous works related to option/sentence level tree search ~\citep{feng2023alphazero, yao2024tree}, it was a common practice to assume that each node in the tree has the same predefined width, \textit{i.e.}, branching factor. This assumption was due to the fact that unlike token-level MCTS with a limited action space, the sample space at the option-level is exceedingly large, with an unlimited number of token combinations. As a result, it was necessary to set a predefined maximum width for each node. However, this predefined branching factor is hard to set, as an improper choice can lead to a search tree that is either too shallow or too thin, resulting in an inefficient exploration of the search space.
% As a result, it was necessary to set a predefined maximum width,  which, however, often leads to an inefficient search space.

% \begin{equation*}
% E_{\phi}(t) = \mathop{\mathbb{E}_{\vo_t^{j}\sim \pi(\vs_t)}}[\min_{\vo_{t}^{i}}|v_{\phi}^{\pi}([\vs_t,\vo_t^{j}])-v_{\phi}^{\pi}([\vs_t,\vo_t^{i}])|]
% \end{equation*}
% \begin{equation*}
% I(\vs_t) = \max_{\vo_{t}^{i}} |v_{\phi}^{\pi}([\vs_t,\vo_t^{i}])- v_{\phi}^{\pi}(\vs_t)|
% \end{equation*}
To quantify the error induced by the branching factor limit, we defined the branching error \(E_{\phi}(t)\). For a node \(t\) with a branching factor of \(m_t\), it aims to use the \(m_t\) child options \(\vo_{t}^{i} \sim \gD_{t}^{children}\) (where \(i \in \{1, \ldots, m_t\}\)) to represent all possible options. Consequently, for a legal option \(\vo_{t}^{j} \sim \pi(\vs_t)\) from the option space, we can calculate the minimal value difference between it and the \(m_t\) existing options, which captures the error associated with representing other possible options using the \(m_t\) available options. It can be formulated as 
$E_{\phi}(t) = \mathop{\mathbb{E}_{\vo_t^{j}\sim \pi(\vs_t)}}[\min_{\vo_{t}^{i}}|v_{\phi}^{\pi}([\vs_t,\vo_t^{j}])-v_{\phi}^{\pi}([\vs_t,\vo_t^{i}])|]$, where $v_{\phi}^{\pi}$ is the value function which will be detailed in \S \ref{sec:critic}. Here we define the importance of node $\vs_t$ as $I(\vs_t) = \max_{\vo_{t}^{i}} |v_{\phi}^{\pi}([\vs_t,\vo_t^{i}])- v_{\phi}^{\pi}(\vs_t)|.$ For simplicity, we assume that the value of the children nodes are uniformly distributed (a detailed analysis of the Gaussian distribution can be found in Appendix \ref{app:node_importance_gaussian}). Under this assumption, we show in Appendix \ref{app:node_importance_uniform} that $E_{\phi}(t) \le \frac{I(\vs_t)}{m_t-1}.$
% \begin{equation*}
% E_{\phi}(t) = \frac{I(\vs_t)}{m_t-1}.
% \end{equation*}
While $E_{\phi}$ is less than some $\epsilon$, we aim to use a smaller total number of nodes for efficiency. 
% This can be formulated as an optimization problem:
% \begin{align*}
% \text{minimize:} & \sum m_t\\
% \text{subject to:} & \frac{I(\vs_t)}{m_t} \le \epsilon
% \end{align*}
\newtheorem{theorem}{Theorem}[section]
\begin{theorem}\label{thm:optimal_branching_factor}
The optimal branching factor $m_t$ in a tree search is set such that $m_t - 1$ is proportional to the node importance $I(\vs_t)$, under the condition $\frac{I(\vs_t)}{m_t-1} \le \epsilon$. \normalfont{Refer to Appendix \ref{app:node_importance_uniform} for the detailed proof.}
\end{theorem}
% \begin{proof}
% See Appendix \ref{app:node_importance_uniform}.
% \end{proof}

A similar concept has also been proposed in ~\cite{taylor2014reinforcement, clouse1996integrating}. Intuitively, $I(\vs_t)$ captures the maximum value deviation from the current state. When this value is small, there is no need to explore further on this node, as there will not be a significant difference by rolling out on this node. Conversely, if the value is large, it is worth trying different children. We set the number of children allowed for a node $n(\vs_t)$ (after extracting $1$) to be linear with this importance, using a factor $\alpha$. In practice, to avoid extreme cases of large variance of $I(\vs_t)$ in the early stage, we bound the number of children by depth-dependent constants $c_{\mathtt{min}}(t)$ and $c_{\mathtt{max}}(t)$, $n(\vs_t) = \max\left(c_{\mathtt{min}}(t), \min\left(\lfloor \alpha I(\vs_t) \rfloor+1, c_{\mathtt{max}}(t)\right)\right).$
% \begin{equation*}
% n(\vs_t) = \max\left(c_{\mathtt{min}}(t), \min\left(\lfloor \alpha I(\vs_t) \rfloor+1, c_{\mathtt{max}}(t)\right)\right).
% \end{equation*}

\subsubsection{State Merge}

% With $n(\vs_t)$ determined, another issue is that options under the same node are very similar, leading to many unnecessary sub-trees. Intuitively, we want the distribution of their values to be spread out. As shown in Appendix \ref{app:node_merge}, given a set of values that are evenly distributed, the worst-case error induced by the branching factor limit is minimized.

% Since we cannot directly control the $\vo_t \sim \pi(\vs_t)$, one strategy to make the value distribution of child nodes more evenly spread is to utilize the concept of move groups ~\cite{van2012revisiting}. By merging similar nodes into the same group, we can potentially influence the minimal value gaps between children. This merging allows us to increase the diversity among groups, thereby covering a larger problem space with limited search rollouts and making the search process more efficient.

% Here, we adapt the definition of node predicate $p_{vM}$ from ~\cite{abel2018state} and ~\cite{fu2024accelerating} to represent whether two nodes are extremely similar. In practice, each time we generate a new option from the policy, we use heuristic functions as $p_{vM}$ to check its similarity with all existing groups. The heuristic function can either be a faster rule-based measurement (e.g., edit distance) or a model-based method (e.g., prompting a language model). Based on this, we decide whether to merge this option with a previous one or create a new group. 

With $n(\vs_t)$ determined, another issue is that options under the same node may be very similar, leading to many unnecessary sub-trees. Since we cannot directly control the $\vo_t \sim \pi(\vs_t)$, one strategy to mitigate this issue is to utilize the concept of move groups, as discussed in ~\cite{van2012revisiting}. By merging similar nodes into the same group, we can increase the diversity among groups, thereby covering a larger problem space with limited search rollouts and making the search process more efficient.

Here, we adapt the definition of node predicate $p_{vM}$ from ~\cite{abel2018state} and ~\cite{fu2024accelerating} to represent whether two nodes are extremely similar. In practice, each time we generate a new option from the policy, we use heuristic functions as $p_{vM}$ to check its similarity with all existing groups. The heuristic function can either be a faster rule-based measurement (e.g., edit distance) or a model-based method (e.g., prompting a language model). Based on this, we decide whether to merge this option with a previous one or create a new group. 

% Another explanation, as shown in Appendix \ref{app:node_merge}, is that given a set of values that are evenly distributed, the worst-case error induced by the branching factor limit is minimized. Intuitively, the state merge can also help make the value distribution of child nodes more evenly spread (although not absolutely, as different states may also have similar values).

\subsubsection{Fast Rollout with Specialized LM}

The simulation operation which employs a rollout policy to project future trajectories from a given state, is crucial for an effective MCTS. This process significantly improves the efficiency of exploration and exploitation, and enhances the accuracy of reward estimation\footnote{Typically, the closer the simulation is to the termination state, the more accurate the reward estimation becomes.}. Estimations made at the end of trajectories tend to have lower bias but higher variance; thus, simulating multiple possible trajectories yields low-bias, low-variance estimates, enabling a more informed and effective search process. Ideally, $\pi_\theta$ would serve as the rollout policy, yet its computational demands render it impractical for the rapid simulations required by MCTS. To address this challenge, we propose the use of a smaller, specialized LM as the fast rollout policy $\pi^{\mathtt{fast}}$. Given a state $\vs_t$, the fast rollout policy $\pi^{\mathtt{fast}}$ efficiently continues generation until it reaches a termination condition, denoted as $\pi^{\mathtt{fast}}(\vs_t)$.

\subsection{Critic}
\label{sec:critic}
% It is crucial for searching algorithms to have reliable guidance signals towards achieving the end goal. 
% In \model{}, we design three types of critic models to guide the search process, \ie value function $v^\pi$ predicting the future reward, process reward models \texttt{PRM} estimating node quality, and outcome reward model \texttt{ORM} assessing the overall trajectory quality. 

In \model{}, we design three types of critic models to guide the search process.

% \paragraph{Value Function} The value function, denoted as $v^\pi(\vs)$, is the expected return starting from state $\vs_t$ following the policy $\pi$ thereafter. To train a value function $v^\pi_\phi(\vs)$ parameterized by $\phi$ ~\citep{feng2023alphazero}, we use the Monte Carlo (MC) estimate to empirically approximate the expected reward by averaging the rewards observed after many samplings starting from state $s$ and following policy $\pi$. Thus, the MC estimate of $v^\pi_\phi(\vs)$ can be written as $v^\pi_\phi(\vs) \approx \frac{1}{J} \sum_{j=1}^{J} G^{(j)}(\vs)$ where $J$ is the number of trajectory starting from state $\vs$, $G^{(j)}(\vs)$ is the total discounted reward from state $s$ in the $j$-th trajectory. Particularly, given the expert demonstration dataset $\gD = \{(\vx_i, \vy_i)\}$, for each prompt $\vx_i$, we generate trajectories $\vtau_i^j = \{\vx_i, \vo_{i1}^j, \vo_{i2}^j, \cdots, \vo_{iT}^j \}$ by following policy $\pi$. A reward $r_i^j$ is assigned to indicate whether $\vtau_i^j$ aligns with $\vy_i$, \eg rewarding trajectories that contains correct answers in mathematical tasks or closely follows the instruction as the ground-truth. We then construct a dataset $\gD_{\mathtt{value}} = \{ (\vs_{it}, v_{it}) | i\in[N], t\in[T]\}$ in which $\vs_{it} = [\vx_i\cdot\vo_{<it}]$ and $v_{it} = \frac{1}{J} \sum_{j=1}^{J} r^j_{iT}$. The value function $v_\phi^\pi$ is optimized by minimizing mean squared error $\gL_\phi = - \sE_{(\vs, v)\sim \gD_{\mathtt{value}}} (v_\phi^\pi(\vs) - v)^2$.
% % \begin{equation*}
% % \gL_\phi = - \sE_{(\vs, v)\sim \gD_{\mathtt{value}}} (v_\phi^\pi(\vs) - v)^2
% % \end{equation*}
% We opt to initialize $v_\phi^\pi$ using the parameters from policy $\pi_\theta$, incorporating an MLP layer on top of it to output a scalar on each token. The scalar prediction at the last token of each state is used as the value.
% %, which can be further written as $G_t^{(i)}(s) = \sum_{k=0}^{\infty} \gamma^k R_{t+k+1}^{(i)}$ in which $R_{t+k+1}^{(i)}$ represents the reward received at time $t+k+1$ in the $i$-th trajectory, and $\gamma$ is the discount factor, with $0 \leq \gamma \leq 1$. We create the 

\paragraph{Value Function} The value function, denoted as $v^\pi(\vs)$, represents the expected return starting from state $\vs$ and following policy $\pi$ thereafter, given by $v^\pi(\vs) = \mathop{\mathbb{E}}_{\tau \sim \pi}[R(\tau)|s_0 = \vs]$ where $R(\tau)$ represents the discounted return of trajectory $\tau$. To train a parameterized value function $v^\pi_\phi(\vs)$, given the prompts $\gD = \{(\vx_i, \cdots) \mid i \in [N]\}$, for each prompt $\vx_i$, we generate multiple trajectories $\vtau_i^j = \{\vx_i, \vo_{i1}^j, \vo_{i2}^j, \cdots, \vo_{iT}^j \}$ by following policy $\pi$ for $J$ times. A final reward $r_i^j$ is assigned to indicate whether $\vtau_i^j$ aligns with $\vy_i$—for example, rewarding trajectories that contain correct answers in mathematical tasks or closely follow instructions as ground truth. We then construct a dataset $\gD_{\mathtt{value}} = \{ (\vs^j_{it}, v^j_{it}) \mid i \in [N], t \in [T], j \in [J] \}$ where $\vs^j_{it} = [\vx_i \cdot \vo^j_{<it}]$ and $v^j_{it} = r^j_i$. The value function $v_\phi^\pi$ is optimized by minimizing the mean squared error: $\gL_\phi = - \sE_{(\vs, v) \sim \gD_{\mathtt{value}}} (v_\phi^\pi(\vs) - v)^2$. Similar to ~\citep{feng2023alphazero}, $v_\phi^\pi$ is a LLM with an MLP layer on top to output a scalar on each token, using the scalar prediction at the last token of each state as the value.

% \begin{equation*}
% \gL_\phi = - \sE_{(\vs, v) \sim \gD_{\mathtt{value}}} (v_\phi^\pi(\vs) - v)^2
% \end{equation*}

\paragraph{PRM} The value function often struggles with credit assignment problem~\citep{sutton1984temporal} and its learning could be inefficient due to delayed and sparse rewards~\citep{sutton2018reinforcement}. Therefore, we propose to incorporate \prm{} that introduces process supervision~\citep{lightman2023let} for direct option assessment. \prm{} generates intrinsic rewards~\citep{chentanez2004intrinsically} to encourage explorations of advantageous options, effectively mitigating issues of reward sparsity by providing immediate, action-specific rewards. Given a state $\vs_t$ and an option $\vo_t$ at time $t$, the \prm{} aims to predict the immediate reward $r_t^{\texttt{PRM}}$ that results from taking option $\vo_t$ in state $\vs_t$. Formally, the \prm{} is a function $R(\vs_t, \vo_t) \rightarrow r^{\mathtt{PRM}}_t$. While \prm{} ideally requires quality labels for each state ~\citep{uesato2022solving}, due to the high cost and time involved in obtaining these, MC estimation with prefix sampling~\citep{wang2023math} is used as a proxy, which aligns with the objective of the value function. Instead of adding a MLP layer on top of the policy model for outputting a scalar reward~\citep{ouyang2022training}, we formulate \prm{} as a text generation task to best leverage LLM's intrinsic knowledge for assessing the quality of an option. We adapt the dataset constructed for the value function as $\gD_{\mathtt{PRM}} = \{ (\vs_{it}, \vo_t, r_t^{\mathtt{PRM}} ) | i\in[N], t\in[T]\}$ where $r_t^{\mathtt{PRM}}$ is the textual description of the reward, \eg an option can be regarded as good if $v_{it}$ is larger than certain threshold. To train \prm{},  we initialize it from the policy model $\pi$ and use the following prompt templates and typical language model loss. The prompt template is shown in Appendix \ref{app:prompt}.

% \begin{tcolorbox}[label=prm_prompt]
% \#\#\#[A detailed rubric that specifies how to evaluate a step of a task]\textbackslash n\textbackslash n\#\#\# State\textbackslash n\{\texttt{state}\}\textbackslash n\textbackslash n\#\#\#Action\textbackslash n\{\texttt{option}\}\textbackslash n\textbackslash n\#\#\#Assessment\textbackslash n\{\texttt{textual reward}\}
% \end{tcolorbox}

\paragraph{ORM} In additional to the value function and \prm{}, \orm{} is also used to guide MCTS. \orm{} is designed to evaluate options sequences in their entirety, assessing the extent to which the complete trajectory aligns with the desired end goal~\citep{uesato2022solving,lightman2023let,wang2023math,feng2023alphazero}. The outcome evaluation complements value function and \prm{} by offering a comprehensive assessment of trajectories. Crucially, \orm{} plays a vital role in the simulation stage of MCTS by providing more accurate signals on the terminal state, which in turn facilitates a more balance between exploration and exploitation strategies. \orm{} is formulated as a text generation task, similar to \prm{}. We leverage the same dataset for the value function training and construct $\gD_{\mathtt{ORM}} = \{ (\vx_i, \vo_{1:T}^i, r_i^{\mathtt{ORM}}) | i\in[N]\}$, where each instance includes a initial state or prompt $\vx_i$, a sequence of actions or options $\vo_{1:T}^i$ taken from that state, and a textual reward $r_i^{\mathtt{ORM}}$ indicating the sequence's success or quality. Similarly, \orm{} is initialized from the policy model $\pi$ and the following prompt templates and language model loss are used for training. The prompt template is shown in Appendix \ref{app:prompt}. \\
% \begin{tcolorbox}
% \#\#\#[A detailed rubric that specifies how to evaluate a complete trajectory of a task]\textbackslash n\textbackslash n\#\#\# Prompt\textbackslash n\{\texttt{prompt}\}\textbackslash n\textbackslash n\#\#\#Trajectory\textbackslash n\{\texttt{trajectory}\}\textbackslash n\textbackslash n\#\#\#Assessment\textbackslash n\{\texttt{textual reward}\}
% \end{tcolorbox}

% provide precision feedback on exploirtation

% \begin{equation*}
% s(\vs) = \beta_{\text{value}} \cdot v_\phi^\pi(\vs) + \beta_{\text{PRM}} \cdot \prm{}(\vs) + \beta_{\text{ORM}} \cdot \mathbb{E}_{\tau \sim \pi^{\mathtt{fast}}(\vs)} [\orm{}(\tau)]
% \end{equation*}

The final score evaluation of a state $\vs$ is a weighted sum of the value function, \prm{}, and \orm{}: $s(\vs) = \beta_{\text{value}} \cdot v_\phi^\pi(\vs) + \beta_{\text{PRM}} \cdot \prm{}(\vs) + \beta_{\text{ORM}} \cdot \mathbb{E}_{\tau \sim \pi^{\mathtt{fast}}(\vs)} [\orm{}(\tau)]$, where $\tau \sim \pi^{\mathtt{fast}}(\vs)$ represents trajectories starting from $\vs$ under $\pi^{\mathtt{fast}}$, and $\beta_{\text{value}}$, $\beta_{\text{PRM}}$, $\beta_{\text{ORM}}$ are hyperparameters. In practice, we found that the value function model has better precision and calibration, while \prm{} has superior recall (Appendix \ref{app:critic_performance}). Although \orm{} with fast rollouts provides low-bias, low-variance estimates, it still inherits some bias from $\pi^{\mathtt{fast}}$. Thus, combining these critics yields a stronger evaluation signal.

\subsection{Policy Self-Improvement}
\label{sec:self_improve}

% We have discussed how \emcts{} can guide policy to find trajectories of higher quality and. In this subsection, we discuss how to leverage these trajectories to further improve the policy. It is an iterative process with each iteration containing two main steps: \emph{data generation} and \emph{policy finetuning}.
The policy improvement an iterative process with each iteration containing two main steps: \emph{data generation} and \emph{policy finetuning}.
\paragraph{Data generation} In this step, we assume to have the current policy $\pi_{\theta_k}$ and synthetic prompts $\gD_k=\{\vx^k_1,\dots\}$ at the $k$-th round, where each $\vx^k_1$ represents a question.
We obtain the corresponding training data $\gD_k$ for policy $\pi_{\theta_k}$ by firstly performing \emcts{} on $\gD_k$ (\S \ref{sec:mcts}) and then sampling a trajectory $\vy^k_i$ from the corresponding tree for each question $\vx^k_i$.
% There are several ways to select a trajectory from a MCTS forest, such as taking a greedy path based on the critic score ($w_i$ in Eq. \ref{eqs:ucb}).
Here we choose the trajectory that yield the highest critic score on the leaf node for each input question.
Next, we filter out instances where the corresponding trajectory is substandard forming $\gD_k = \{(\vx^k_i, \vy^k_i)~|~f(\vx^k_i, \vy^k_i)>\gamma\}$
% \begin{equation*}
%  \gD_k = \{(\vx^k_i, \vy^k_i)~|~f(\vx^k_i, \vy^k_i)>\gamma\}
% \end{equation*}
where $f$ represents a function for quality scoring, and $\gamma$ indicates a threshold.
There can be several ways to implement the function, and here we simply use the \orm{} (\S \ref{sec:critic}).

% In this step, we assume to have the current policy $\pi_{\theta_k}$, value network $v^{\pi}$, \orm{} and synthetic prompts $\gD=\{\vx^k_1,\dots\}$, where each $\vx^k_1$ represents a question.
% We obtain the corresponding training data $\gD_k$ for policy $\pi_{\theta_k}$ by first performing MCTS on $D$ (\S \ref{sec:mcts}) and then sampling a trajectory $\vy^k_i$ from the corresponding MCTS forest for each question $\vx^k_i$.
% There are several ways to select a trajectory from a MCTS forest, such as taking a greedy path based on the critic score ($w_i$ in Eq. \ref{eq:mcts}).
% Here we choose the trajectory that yield the highest critic score on the leaf node for each input question.
% As the next step, we filter out instances where the corresponding trajectory is not in high quality:


% consider whether the the final answer of $y^i$ correct (equals to $a^i$).

\paragraph{Policy finetuning}
% With the obtained training data $\gD_k$, we organize the data into the following prompt templates $P_{SI}$:
With the obtained training data $\gD_k$, we organize the data into the prompt templates shown in Appendix \ref{app:prompt}. Then the policy $\pi_{\theta_k}$ is finetuned using target-loss: $\mathcal{L}_{\theta_k} = \mathbb{E}_{(\vx^k_i, \vy^k_i) \sim \gD_k} \big[\log \pi_{\theta_k}(\vy^k_i|\vx^k_i) \big]$, resulting in an updated policy $\pi_{\theta_{k+1}}$. We leave other training methods, such as DPO \citep{rafailov2023direct} or PPO \citep{schulman2017proximal} in future work.
% \begin{equation*}
%  \mathcal{L}_{\theta_k} = \mathbb{E}_{(\vx^k_i, \vy^k_i) \sim \gD_k} \big[\log \pi_{\theta_k}(\vy^k_i|\vx^k_i) \big]
% \end{equation*}
% This results in an updated policy $\pi_{\theta_{k+1}}$.
% We leave other training methods, such as DPO \citep{rafailov2023direct} or PPO \citep{schulman2017proximal} in future work.
% With the new policy, the value network and ORM is then updated as described in \S \ref{}.

\section{Experiments}
\label{sec:exp}


% \subsection{Evaluation Setups}

\subsection{Experiment Setups}

\model{} is generally applicable to a wide spectrum tasks. As an early exploration, in this paper, we conduct experiments on mathematical reasoning problems where the learning signals are clear to define \ie, final answer is correct or wrong. We choose to evaluate on two widely used datasets GSM8K~\citep{gsm8k} and MATH~\citep{math}. For GSM8K, we utilize the whole test set while for MATH, due to computation constraints, we utilize a subset following the same procedure of~\cite{lightman2023let}. We evaluate the performance of predicting answers correctly for policy models. In addition, we calculate the average rollouts, represented by the number of nodes in the tree, as a measure of computational efficiency. We compare the performance of \model{} with a suite of proprietary model, including OpenAI's GPT-4 and GPT-3.5, Anthropic's Claude-2, as well as Google's PaLM-2 and the gemini model family. To ensure a fair and consistent evaluation, we employ CoT as our primary prompting method. Additionally, we conduct comparisons with strong open-source models, including Llama-2-70b~\citep{llama2} and WizardMath-70B-V1.0~\citep{wizardmath}. % For LLaMA-2 70B, we present results from few-shot prompting as well as zero-shot prompting for its SFT version, which was trained using CoT rationales and final answers. Wizardmath 70B has been trained on a diverse set of mathematical data generated by ChatGPT, employing both SFT and RLHF. We provide zero-shot prompting results.

%Means square errors and next token prediction accuracy are reported for evaluating value functions and \prm{}/\orm{}, respectively.
% \paragraph{Baseline Systems} We evaluate the performance of \model{} against a suite of proprietary model, including OpenAI's GPT-4 and GPT-3.5, Anthropic's Claude-2, as well as Google's PaLM-2 and the gemini model family. To ensure a fair and consistent evaluation, we employ CoT as our primary prompting method. We additionally report PAL~\citep{gao2023pal} prompting performance with GPT-4 as it demonstrates enhanced performance. Additionally, we conduct comparisons with strong open-source models, including LLaMA-2 70B~\citep{llama2} and Wizardmath 70B~\citep{wizardmath}. For LLaMA-2 70B, we present results from few-shot prompting as well as zero-shot prompting for its SFT version, which was trained using CoT rationales and final answers. Wizardmath 70B has been trained on a diverse set of mathematical data generated by ChatGPT, employing both SFT and RLHF. We provide zero-shot prompting results.

% The implementation details can be found in Appendix \ref{app:implementation}.


% \paragraph{Datasets} \model{} is generally applicable to a wide spectrum tasks. As an early exploration, in this paper, we conduct experiments on mathematical reasoning problems where the learning signals are clear to define \ie, final answer is correct or wrong. We choose to evaluate on two widely used datasets GSM8K~\citep{gsm8k} and MATH~\citep{math}. For GSM8K, we utilize the whole test set while for MATH, due to computation constraints, we utilize a subset following the same procedure of~\cite{lightman2023let}.
% \paragraph{Metrics} We evaluate the performance of predicting answers correctly for policy models. In the same time, we calculate the average rollouts, represented by the number of nodes in the tree, as a measure of computational efficiency. %Means square errors and next token prediction accuracy are reported for evaluating value functions and \prm{}/\orm{}, respectively.
% \paragraph{Baseline Systems} We evaluate the performance of \model{} against a suite of proprietary model, including OpenAI's GPT-4 and GPT-3.5, Anthropic's Claude-2, as well as Google's PaLM-2 and the gemini model family. To ensure a fair and consistent evaluation, we employ CoT as our primary prompting method. We additionally report PAL~\citep{gao2023pal} prompting performance with GPT-4 as it demonstrates enhanced performance. Additionally, we conduct comparisons with strong open-source models, including LLaMA-2 70B~\citep{llama2} and Wizardmath 70B~\citep{wizardmath}. For LLaMA-2 70B, we present results from few-shot prompting as well as zero-shot prompting for its SFT version, which was trained using CoT rationales and final answers. Wizardmath 70B has been trained on a diverse set of mathematical data generated by ChatGPT, employing both SFT and RLHF. We provide zero-shot prompting results.

% The implementation details can be found in Appendix \ref{app:implementation}.
% \subsection{Baseline Systems}
% We evaluate the performance of \model{} against a suite of proprietary model, including OpenAI's GPT-4 and GPT-3.5, Anthropic's Claude-2, as well as Google's PaLM-2 and the gemini model family. To ensure a fair and consistent evaluation, we employ CoT as our primary prompting method. We additionally report PAL~\citep{gao2023pal} prompting performance with GPT-4 as it demonstrates enhanced performance.

% Additionally, we conduct comparisons with strong open-source models, including LLaMA-2 70B~\citep{llama2} and Wizardmath 70B~\citep{wizardmath}. For LLaMA-2 70B, we present results from few-shot prompting as well as zero-shot prompting for its SFT version, which was trained using CoT rationales and final answers. Wizardmath 70B has been trained on a diverse set of mathematical data generated by ChatGPT, employing both SFT and RLHF. We provide zero-shot prompting results.

% We conduct our experiments on two math datasets, GSM8K~\citep{gsm8k} and MATH~\citep{math}.
% Our use LLaMa 2 70B~\citep{llama2} and WizardMath 70B V1.0~\citep{wizardmath} as our base models for GSM8K and MATH respectively.

% \subsection{Implementation Details}

We select Llama-2-70b as the policy model for the GSM8K dataset and WizardMath-70B-V1.0 for the MATH dataset. To construct the training dataset for the value function, \prm{} and \orm{}, we generate 50 trajectories for each prompt and construct the training target following Section~\ref{sec:critic}. Both \prm{} and \orm{} are initialized using the weights from the policy model, while the value function uses a smaller Llama-2-13b model, as we observed no performance gains from increasing the value function model size. In the design of \orm{}, tool usage is not incorporated for GSM8K. However, for MATH, we enhance \orm{} by incorporating tools like python sympy to assess the quality of a trajectory, in a manner similar to that described by \citet{gou2023tora}. The training employ a learning rate of 1e-6 and are trained for one epoch. For the fast rollout policy model, we opt for the Abel-002-7B model~\citep{abel} for both the GSM8K and MATH tasks for its high efficiency and superior performance. For the MCTS parameters, they are configured at different scales, as shown in Appendix \ref{app:implementation}. We set $\beta_{\text{value}}$, $\beta_{\text{PRM}}$, and $\beta_{\text{ORM}}$ all to 1.0.

% We set the MCTS parameters as follows: in GSM8K, $c=1$ for the small scale (\texttt{\#rollout}) and $1.5$ for the large scale, with $\alpha=1$. For $t=0$, $c_\text{min}(0)=10$ for the small scale and $40$ for the large scale, while for the rest of $t$, $c_\text{min}(t)=2$. We also set $c_\text{max}(0) = 10$ for the small scale and $40$ for the large scale, and for the remaining $t$, $c_\text{max}(t)=10$. The termination condition is based on sentence termination. In MATH, the parameters are $c=1$, $\alpha=1$, and for $t=0$, $c_\text{min}(0)=10$ for the small scale and $20$ for the large scale, while for the rest of $t$, $c_\text{min}(t)=3$. We set $c_\text{max}(0) = 10$ for the small scale and $20$ for the large scale, and for the remaining $t$, $c_\text{max}(t)=10$. The termination function is rule-based, checking if there are any formulations or calculations in the sentence. If there are, the option is terminated; otherwise, the option continues to extend.

For policy self-improving (\S \ref{sec:self_improve}), we train the policy model up to 3 epochs, setting batch size to 128, learning rate to $5\times 10^{-6}$ and minimal learning rate to $1\times 10^{-6}$.
Linear warm-up and decay is used with warm-up percent to be 10\%.
We perform early stopping based on a devset held out from the training instances.
For GSM8K experiments, we perform two rounds of self-improving, synthesizing 6.4k and 7.9k prompts\citep{yu2023metamath} respectively to obtain the corresponding MCTS outputs for training.
For MATH experiments, we only perform one round of self-improving due to limited computation resources, and 5.9k prompts are synthesized.

The termination function for options can be either be learned or rule-based. In practice, for the GSM8K dataset, the termination condition occurs at the end of each line. This is based on the typical structure of this dataset, where each line represents a distinct step or point. For the MATH dataset, due to its complexity and the base model's tendency to generate many \texttt{\textbackslash n\textbackslash n} line breaks with some less meaningful content between them, termination occurs at the end of a line if a formula pattern is detected. During inference, if \texttt{\textbackslash n\textbackslash n} is encountered, we perform a rule-based check for formula patterns. It terminates if a pattern is found or continues generating until the next \texttt{\textbackslash n\textbackslash n}.


\subsection{Results}
% Compare all existing approaches, add MATH results
% \begin{table}[!htb]
%     \centering
%     \begin{tabular}{ll|c|c}
%         \multicolumn{2}{l|}{Method}         & GSM8K  & MATH   \\
%         \hline
%         \multicolumn{2}{l|}{GPT-4 }         & $92.0$ & $42.5$ \\
%         \multicolumn{2}{l|}{GPT-4 (PAL)}    & $94.2$ & $51.8$ \\
%         \hline
%          \multirow{3}{*}{Gemini} & 1.0 Pro  & $77.9$ & $32.6$ \\
%          & 1.0 Ultra                        & $88.9$ & $53.2$ \\
%          & 1.5 Pro                          & $92.5$ & $58.5$ \\
%         \hline
%         \multicolumn{2}{l|}{ChatGPT}        & $80.8$ & $35.5$ \\
%         \multicolumn{2}{l|}{Claude-2}       & $85.2$ & $32.5$ \\
%         \multicolumn{2}{l|}{PaLM-2}         & $80.7$ & $34.3$ \\
%         \hline
%         \multicolumn{2}{l|}{LLaMA-2 70B}     & $57.8$ & $14.4$ \\
%         \multicolumn{2}{l|}{LLaMA-2 70B SFT} & $69.3$ & $14.9$ \\
%       % \multicolumn{2}{l|}{ToRA}            & $84.3$ & $49.7$ \\
%     \multicolumn{2}{l|}{WizardMath 70B V1.0} & $81.6$ & $22.7$ \\
%         \hline \hline
%         \multirow{2}{*}{MCTS} & Base model  & $88.9$ & $48.7$ \\
%          & Improved policy                  & $92.4$ & $51.0$   \\
%         \hline
%     \end{tabular}
%     \vspace{4mm}
%     \caption{Overall results on GSM8K and MATH test sets. 
%     We use LLaMA-2 70B and WizardMath 70B V1.0 as our base models on GSM8K and MATH data sets respectively. 
%     }
%     \label{tab:final_result}
% \end{table}

{
\renewcommand{\arraystretch}{1.05}
% \begin{table*}[!t]
% \centering
% \scalebox{1.1}{    
% 	\setlength\tabcolsep{6pt}
% 	% \begin{threeparttable}
% 		% \fontsize{9}{9}
% 		% \selectfont
% 		\begin{tabular}{lcc|cc}
% 			\toprule
% 			Model         & \texttt{IDD} & \texttt{SYN} & \texttt{GSM8K} & \texttt{MATH} \cr 
% 			\midrule
%     GPT-3.5~\cite{} & - & - & 80.8 & 35.5 \cr
%    GPT-4~\cite{} & - & - & 92.0 & 42.5 \cr
%    GPT-4 (PAL)~\cite{} & - & - & 94.2 & 51.8 \cr
%    \midrule
%    Gemini 1.0 Pro~\cite{} & - & - & 77.9 & 32.6 \cr
%    Gemini 1.0 Ultra~\cite{} & - & - & 88.9 & 53.2 \cr
%    Gemini 1.5 Pro~\cite{} & - & - & 92.5 & 58.5 \cr
%    \midrule
%    Claude-2~\cite{} & - & - & 85.2 & 32.5 \cr
%    PaLM-2 540B~\cite{} & - & - & 80.7 & 34.3 \cr
%    \midrule
%    LLaMA-2 70B & $\times$ & $\times$ & 57.8 & 14.4 \cr
%    LLaMA-2 70B SFT & $\checkmark$ & $\times$ & 69.3 & 14.9 \cr
%    WizardMath 70B V1.0 & $\times$ & $\times$ & 81.6 & 22.7 \cr
%    \midrule
%    \model{} & $\checkmark$ & $\times$ & 88.9 & 48.7 \cr
%    \model{} & $\checkmark$ & $\checkmark$ & 92.4 & 51.0 \cr
% 			\bottomrule  
% 		\end{tabular}
% 	% \end{threeparttable}
% 		  }
% 	\caption{Overall results on GSM8K and MATH test sets. 
%     We use LLaMA-2 70B and WizardMath 70B V1.0 as our base models on GSM8K and MATH data sets respectively. \texttt{IDD} indicates that the model has been trained using in-domain data. \texttt{SYN} denotes that the model has been trained on synthetic prompts, with trajectories generated using MCTS. }
% 	\label{table:main_results}
% \end{table*}

\begin{table*}[!t]
\small
    \centering
    % \scalebox{1.1}{    
        \setlength\tabcolsep{6pt}
        % \begin{threeparttable}
        % \fontsize{9}{9}
        % \selectfont
        \begin{tabular}{lccccc|cc}
            \toprule
            Model                    & \texttt{Decoding} & \texttt{\#Annotation} & \texttt{RN} & \texttt{FA} & \texttt{SYN} & \texttt{GSM8K} & \texttt{MATH} \cr 
            \midrule
            GPT-3.5~\cite{}          & Sampling & - & - & -             & -            & 80.8           & 35.5 \cr          
            GPT-4~\cite{}            & Sampling & -  & - & -          & -            & 92.0           & 42.5 \cr          
            GPT-4 (PAL)~\cite{}      & Sampling & -   & - & -         & -            & 94.2           & 51.8 \cr          
            \midrule
            Gemini 1.0 Pro~\cite{}   & Sampling & -   & - & -          & -            & 77.9           & 32.6 \cr          
            Gemini 1.0 Ultra~\cite{} & Sampling & -    & - & -         & -            & 88.9           & 53.2 \cr          
            Gemini 1.5 Pro~\cite{}   & Sampling & -     & - & -        & -            & 92.5           & 58.5 \cr          
            \midrule
            Claude-2~\cite{}         & Sampling & -     & - & -        & -            & 85.2           & 32.5 \cr          
            PaLM-2 540B~\cite{}      & Sampling & -      & - & -       & -            & 80.7           & 34.3 \cr          
            \midrule
            Llama-2-70b              & Greedy & 0 & $\times$ & $\times$ & $\times$         & 57.8           & - \cr          
            Llama-2-70b SFT          & Greedy & 7.5k & $\checkmark$ & $\checkmark$ & $\times$     & 69.3           & - \cr          
            WizardMath-70B-V1.0      & Greedy & 96k & $\checkmark$ & $\checkmark$ & $\times$         & -           & 20.7 \cr          
            \model{}                 & Greedy & 7.5k/7.5k & $\times$ & $\checkmark$ & $\checkmark$ & 73.7           & 23.6 \cr         
            \midrule
            \model{}                 & \emcts{} & 7.5k/7.5k & $\times$ & $\checkmark$ & $\times$      & 88.9           & 48.7 \cr          
            \model{}                 & \emcts{} & 7.5k/7.5k & $\times$ & $\checkmark$ & $\checkmark$  & 92.0           & 51.0 \cr                       
            \bottomrule   
        \end{tabular}
        % \end{threeparttable}
    
    % \caption{Comparison results of \model{} on the GSM8K and MATH datasets, utilizing LLaMA-2 70B and WizardMath 70B V1.0 as base models for GSM8K and MATH datasets, respectively. \texttt{\#Annotation} indicates the quantity of labeled data employed for fine-tuning each base model. The annotation used for training are noted as \texttt{RN} for rationales and \texttt{FA} for final answers. \texttt{SYN} means models trained on synthetic prompts, where trajectories were generated using \emcts{}. }

\caption{Comparison results of \model{} on the GSM8K and MATH datasets. \texttt{\#Annotation} indicates the quantity of labeled data employed for fine-tuning policy or training critic models. The annotation used for training are noted as \texttt{RN} for rationales and \texttt{FA} for final answers. \texttt{SYN} means models trained on synthetic prompts, where trajectories were generated using \emcts{}. }
    
    \label{table:main_results}
\end{table*}
}
% Table~\ref{table:main_results} presents the experimental results of \model{} on GSM8K and MATH. We observe that

Table~\ref{table:main_results} lists the performance comparisons of various methods on the GSM8K and MATH datasets. Our findings reveal that \model{}, based on Llama-2-70B and WizardMath-70B-V1.0, utilizes only final answer annotations and continues to improve through training on responses from \emcts{}. This comparison underscores the efficacy and broad applicability of our imagination-searching-criticizing self-improving framework. Moreover, when our model is augmented with \emcts{} decoding strategy, its performance markedly improves, achieving scores of 88.9 and 48.7 on the GSM8K and MATH datasets, respectively. Following two iterations of self-improvement using synthetic prompts, \model{} demonstrates performance comparable to that of GPT-4. This suggests a viable approach to improving LLMs' capabilities in complex problem-solving tasks in a self-improving fashion, leveraging a minimal amount of labeled data. We also analyze the performance of various search methods in Appendix \ref{app:search_comparison}.

% In addition, table~\ref{table:search_comparison} presents the performance of various methods applied to different number of responses, from 10 to 50. Our analysis confirms several key findings: 1) Reranking utilizing \orm{} consistently outperforms self-consistency techniques, indicating that \orm{} is capable of generating meaningful signals for searching. 2) \emcts{} demonstrates superior performance while requiring significantly fewer rollouts. For instance, on the MATH dataset, \emcts{} achieves better results with only half the number of rollouts compared to reranking. These results suggest that our design of an efficient MCTS in \model{} can serve as an effective policy improvement operation, enabling the search for high-quality trajectories with reduced computational cost.


\subsection{Ablation Study}
% % baseline (sampling)
% self consistence w/ diff. size of n-samples
% reranking w/ diff. size of n-samples
% MCTS w/ diff. size of rollout
% \begin{table}[!htb]
%     \centering
%     \setlength{\tabcolsep}{4pt}
%     \begin{tabular}{c||c|c|c||c|c|c}
%         \multirow{2}{*}{Method}           & \multicolumn{3}{c||}{GSM8K} &   \multicolumn{3}{c}{MATH} \\
%         \cline{2-7}
%             & \# of outputs & \# of rollouts  & Accuracy & \# of outputs & \# of rollouts  & Accuracy \\
%         \hline \hline
%         Baseline                          & $1$  & $4.6$ & $57.8$ & $1$ & $9.9$ & $20.7$ \\
%         \hline\hline
%         \multirow{3}{*}{Self-consistence} & $10$ & $46$  & $67.4$ &  $10$   &  $99$   & $22.5$ \\
%                                           & $30$ & $137$ & $74.2$ &  $30$   &  $299$  & $27.3$ \\
%                                           & $50$ & $229$ & $75.4$ &  $50$   &  $499$  & $28.8$ \\
%         \hline\hline
%         % MATH ORM V4 results: tool use
%         \multirow{3}{*}{Re-ranking}       & $10$ & $46$  & $80.8$ &  $10$   &  $99$   &  $34.1$ \\
%                                           & $30$ & $137$ & $86.3$ &  $30$   &  $299$  &  $39.0$ \\
%                                           & $50$ & $229$ & $87.7$ &  $50$   &  $499$  &  $42.0$ \\
%         % MATH ORM V2 results: no tool use for ORM
%         % \multirow{3}{*}{Re-ranking}       & $10$ & $46$  & $80.8$ & $10$    &  $99$  & $26.0$ \\
%         %                                   & $30$ & $137$ & $86.3$ &  $30$   &  $299$  & $27.3$ \\
%         %                                   & $50$ & $229$ & $87.7$ &  $50$   &  $499$  & $27.9$ \\
%         \hline\hline
%         \multirow{2}{*}{MCTS}             & N/A & $55$   & $87.0$ &  N/A   &  $223$  & $45.4$ \\
%                                           & N/A & $230$  & $88.9$ &  N/A   &  $341$  & $48.7$ \\
%         \hline
%     \end{tabular}
%     \vspace{4mm}
%     \caption{MCTS results over GSM8K and MATH test sets. We use LLaMA-2 70B and WizardMath 70B V1.0 as our base models on GSM8K and MATH data sets respectively.
%     *: we test WizardMath 70B V1.0 model with~\protect\hyperlink{https://github.com/FastEval/FastEval}{FastEval} script, 
%     which is also used for all our methods in order to have an apple to apple comparison.}
%     \label{tab:mcts_result}
% \end{table}

{
\renewcommand{\arraystretch}{1.0}
\begin{table*}[!t]
\centering
% \scalebox{1.0f}{    
	% \setlength\tabcolsep{3pt}
	% \begin{threeparttable}
		% \fontsize{9}{9}
		% \selectfont
		\begin{tabular}{lc|cc|cc}
			\toprule
			\multirow{2}{*}{Method} & \multirow{2}{*}{\#Responses} &  \multicolumn{2}{c}{GSM8K} & \multicolumn{2}{c}{MATH} \cr
   \cmidrule(lr){3-4} \cmidrule(lr){5-6}

    & & \texttt{\#Rollouts} & \texttt{Accuracy} & \texttt{\#Rollouts} & \texttt{Accuracy} \cr
   
   \midrule
   Greedy                         & 1  & 4.6 & 57.8  & 9.9 & 20.7 \\
\midrule    
   \multirow{3}{*}{Self-consistency} & 10 & 46  & 67.4    &  99   & 22.5 \\
& 30 & 137 & 74.2    &  299  & 27.3 \\
& 50 & 229 & 75.4   &  499  & 28.8 \\
\midrule
  \multirow{3}{*}{Re-ranking}       & 10 & 46  & 80.8 &    99   &  34.1 \\
                                          & 30 & 137 & 86.3 &  299  &  39.0 \\
                                          & 50 & 229 & 87.7 &    499  &  42.0 \\
\midrule
\multirow{2}{*}{\emcts{}}             & - & 55   & 87.0 &   223  & 45.4 \\
                                          & - & 230  & 88.9 &   341  & 48.7 \\
    
			\bottomrule  
		\end{tabular}
	% \end{threeparttable}
		  
    
	% \caption{MCTS results over GSM8K and MATH test sets. We use LLaMA-2 70B and WizardMath 70B V1.0 as our base models on GSM8K and MATH data sets respectively.*: we test WizardMath 70B V1.0 model with~\protect\hyperlink{https://github.com/FastEval/FastEval}{FastEval} script, which is also used for all our methods in order to have an apple to apple comparison.}
 \caption{Comparative results of various searching method on GSM8K and MATH.}
	\label{table:search_comparison}
 
\end{table*}
}

% Best-of-1 accuracy: 15.09%,  n_rollout 24
% Best-of-5 accuracy: 24.74%,  n_rollout 124
% Best-of-10 accuracy: 26.00%,  n_rollout 249
% Best-of-20 accuracy: 26.84%,  n_rollout 499
% Best-of-30 accuracy: 27.30%,  n_rollout 749

% Analysis: 1. acc. vs \# of rollouts

% \begin{figure}[htbp]
    \centering
    \includegraphics[width=0.75\textwidth]{figures/mcts_ablation.png}
    \caption{Ablation study on the GSM8K test set of various enhancements to the proposed efficient MCTS, including \prm{}, fastrollout with \orm{}, state merging, and increasing the number of rollouts.}
    \label{fig:search_ablation}
\end{figure}


% \begin{figure}[ht]
%     \centering
%     % Minipage for the Table
%     \begin{minipage}{0.25\textwidth}
%         \centering
%         \begin{tabular}{|l|c|}
%             \hline
%             Column 1 & Column 2 \\
%             \hline
%             Item 1 & Item 2 \\
%             Item 3 & Item 4 \\
%             \hline
%         \end{tabular}
%         \caption{Example Table}
%         \label{table:example_table}
%     \end{minipage}
%     \hfill
%     % Minipage for the Figure
%     \begin{minipage}{0.7\textwidth}
%         \centering
%         \includegraphics[width=0.9\textwidth]{figures/mcts_ablation.png}
%     \caption{Ablation study on the GSM8K test set of various enhancements to the proposed efficient MCTS, including \prm{}, fastrollout with \orm{}, state merging, and increasing the number of rollouts.}
%     \label{fig:search_ablation}
%     \end{minipage}
% \end{figure}

\begin{table}[h]
\small
\centering
\begin{minipage}{.5\linewidth}
\centering

\begin{tabular}{ccccc|c}
\toprule
\texttt{AB} & \prm{} & \texttt{FR}-\orm{} & \texttt{SM} & \texttt{LG-\#Rollout} & Acc \cr
\midrule
$\times$ & $\times$ & $\times$ & $\times$ & $\times$ & 79.5 \\
$\checkmark$ & $\times$ & $\times$ & $\times$ & $\times$ & 84.9 \\
$\checkmark$ & $\checkmark$ & $\times$ & $\times$ & $\times$ & 85.9 \\
$\checkmark$ & $\checkmark$ & $\checkmark$ & $\times$ & $\times$ & 86.5 \\
$\checkmark$ & $\checkmark$ & $\checkmark$ & $\checkmark$ & $\times$ & 87.0 \\
$\checkmark$ & $\checkmark$ & $\checkmark$ & $\checkmark$ & $\checkmark$ & 88.9 \\
\bottomrule
\end{tabular}
\vspace{2mm}
\caption*{(a) Ablation study on GSM8K}
% \caption{First Table: Ablation study on GSM8K test set.}
\end{minipage}%
\begin{minipage}{.5\linewidth}
\centering

\begin{tabular}{cc|cc}
\toprule
\texttt{TA}-\orm{} & \texttt{Option}  & \texttt{Acc} & \texttt{\#Rollout} \cr
\midrule
$\times$ & $\times$ & 38.8 & 201 \\
$\checkmark$ & $\times$ & 44.1 & 198 \\
$\checkmark$ & $\checkmark$ & 45.4 & 148 \\
\bottomrule
\end{tabular}
\vspace{2mm}
\caption*{(b) Ablation study on MATH}
\end{minipage}
\caption{\textbf{(a)}: Ablation studies on the GSM8K test set of various components of \emcts{}, including adaptive branching, \prm{}, fast-rollout with \orm{}, state merge, and large number of rollouts. \textbf{(b)}: Ablation studies of the impacts of tool-augmented \orm{} and option-level formulation  on MATH.}
\label{table:ablation}
\end{table}


We assess the effectiveness of each component in \model{} and report the results on GSM8K in Table~\ref{table:ablation}(a). Vanilla MCTS, configured with only the value function and a fixed number of children per node, achieves an accuracy of 79.5\%. This serves as a reference point for evaluating the incremental benefits introduced by each additional component. The use of adaptive branching increae the accuracy to 84.9\%. The addition of \prm{} improves the accuracy modestly to 85.9\%, showing the effectivenss of process supervision for searching. A more significant improvement is observed with the introduction of \orm{} with fast rollout, which boosts the accuracy to 86.5\%.  Integrating state merging results in a further increase in accuracy, reaching 87.0\%. Finally the combined of increasing the number of rollouts with the other components yields the best performance on this task. 

Table~\ref{table:ablation}(b) presents the ablation study of option formulation and the tool-augmented critic on the MATH dataset. Our proposed \emcts{} achieves an accuracy of 45.4 with 148 rollouts. When options are excluded, reverting to essentially sentence-level MCTS, the performance decreases to 44.1 with a noticeable increase in the number of rollouts to 198. This demonstrates that option formulation introduces enhanced flexibility to MCTS, enabling better performance with fewer search efforts. Furthermore, the most significant decrease in performance is observed when only intrinsic knowledge is utilized for \orm{}, which drops to an accuracy of 38.8. This suggests that the absence of an external tool critically impedes the \orm{}'s capability to effectively assess challenging math problems.


% \begin{wraptable}{r}{5.5cm}
% \label{tab:option_critic}
% % \vspace{-5mm}

% % \label{tab:my_label}
% \begin{tabular}{lcc}

% \toprule
% Method & \texttt{Acc} & \texttt{\#Rollout} \\
% \midrule
% \emcts{} & 45.4 & 148 \\
% \ w/o option & 44.1 & 198 \\
% \ \orm{} w/o tool  & 38.8 & 201 \\
% \bottomrule

% \end{tabular}
% \caption{Comparison results of options formulation on MATH.}
% \end{wraptable}
\begin{figure}[!tbp]
    \centering
    \includegraphics[width=0.9\textwidth]{figures/model_self_improving_n_rounds_results_v2.png}
    \caption{Empirical analysis on GSM8K of different self-improving data collection methods and number of iterations. Models are evaluated with greedy decoding, \emcts{} with small \#rollout and large \#rollout. }
    \label{fig:self_improving_ablations}
\end{figure}
Figure~\ref{fig:self_improving_ablations} depicts a comparative results on GSM8K of two rounds of self-improving trained on trajectories collected using reranking and \emcts{}. We report the performance of greedy decoding, \emcts{} with a relatively small number of rollouts (50-60), and \emcts{} with a larger number of rollouts (200-300) for each model. We observe that 1) Models trained on the trajectories from reranking or \emcts{} outperform the initial policy by a significant margin. In addition, the performance can be iteratively improved with training suggesting that self-improving has the potential to achieve continual performance gain. 2) While both reranking and \emcts{} can generate high-quality trajectories for self-improving , \emcts{} is performant with high efficiency and better accuracy. Models trained on trajectories generated by it not only exceed the performance of those trained on reranked trajectories but also, when decoded with \emcts{}, demonstrate on par performance with GPT-4, revealing that \model{} is an effective self-improving framework.

% 2) ablation study:
% search times, each components


\begin{table}[h]
\small
\centering
\begin{minipage}{.45\linewidth}
\centering

    \begin{tabular}{cl|c|c}
    \toprule
        \multicolumn{2}{c|}{\texttt{Method}}         & \texttt{Threshold}  & \texttt{Acc}\\
        % \hline
        \midrule
           &   Edit distance	               & $20$ & $86.8$ \\
           &   Edit distance	             & $50$ & $87.0$\\
           &  Cosine Similarity	            & $0.7$ & $86.3$\\
           & Model-based	& N/A	& $86.7$ \\
        \bottomrule
    \end{tabular}
\vspace{2mm}
\caption*{(a) Ablation on the choice of state merge functions.}
% \caption{First Table: Ablation study on GSM8K test set.}
\end{minipage}%
\begin{minipage}{.55\linewidth}
\centering

    \begin{tabular}{cl|c}
    \toprule
        \multicolumn{2}{c|}{\texttt{\#Trajetory}}         & \texttt{Acc}\\
        \midrule
           &   $1$	                & $85.9$ \\
           &   $4$	           & $86.5$\\
           &  $8$	       & $86.7$\\
        \bottomrule
    \end{tabular}
\vspace{2mm}
\caption*{(b) Ablation on the number of trajectories.}
\end{minipage}
\caption{\textbf{(a)}: Ablation studies on the choice of heuristic/model-based functions in state merge on GSM8K with base Llama2-70b. The model used in the model-based state merge is Llama-2-70b-chat. \textbf{(b)}: Ablation studies of the number of rollout trajectories in fast-rollout estimation on GSM8K with base Llama2-70b.}
\label{table:ablation_sm}
\end{table}

We further analyze the impact of different hyperparameters and design choices for each component. Table~\ref{table:ablation_sm}(a) shows that varying heuristic functions (with hyperparameters) for state merge has limited impact on performance. Table~\ref{table:ablation_sm}(b) shows that, as the number of fast-rollouts increases, there is a corresponding improvement in performance. This is due to the reduction in the variance of the estimates. We used $n=4$ in our experiments for better trade-off between performance and efficiency. Additional ablations on the choice of fast-rollout models, are provided in Appendix \ref{app:add_ablations}.


% Analysis: 
% x: bin of acc.  y: avg step/outcome reward
% value acc. for fast-rollout vs no fast-rollout
% Tool use vs. no tool use on MATH set (v$_2$ small vs. v$_4$ small)

% \subsection{Self-improving Results}





% \section{Limitations and Future Work}
% Despite the promising results demonstrated by \model{} in this study, there are several limitations that requires further exploration. (\RN{1}) Our current implementation employs relatively simple methods for generating synthetic prompts. Future iterations of \model{} should explore advanced techniques, such as Self-Instruct, to create both diverse and model capability-awared prompts. (\RN{2}) Although \model{} demonstrates improvements over base models, its performance in greedy sampling is substantially inferior to that observed when decoded with \emcts{}. This indicates that the full potential of MCTS for self-improvement in LLMs has not yet been fully realized. Two potential factors contributing to this issue have been identified: a) the self-improvement loop may not be leveraging sufficient data; and b) the base model may be limited in its capacity for rapid learning. Addressing these concerns could lead to more significant improvemens. (\RN{3}) In our existing framework, the critic models remain static. We will explore mechanisms to continually update critic models to adapt to new policy models. This will help ensure the discriminator-generator gap and improve the overall training dynamics. (\RN{4}) The evaluation of \model{} has been limited to mathematical reasoning tasks. To verify the generalizability and broader applicability of the framework, future research will need to extend its application to other domains.

\section{Conclusion}
\label{sec:con}
In this paper, we introduce \model{}, an imagination-searching-criticizing framework designed for the self-improvement of LLMs without the necessity of additional annotations. At the heart of it is the integration of MCTS with LLMs. To tackle the inherent challenges associated with this integration, including data scarcity, the vastness of search spaces, and the subjective nature of feedback in language tasks, we introduce a data synthesizer for strategic prompt synthesis, an optimized MCTS tailored for efficient search in language tasks, and a trio of critic models to provide precise feedback. Our experimental findings on mathematical reasoning tasks reveal that \model{} significantly boosts the performance of LLMs without requiring extra data annotations. Moreover, when decoded with \emcts{}, \model{} performs comparably to GPT-4, highlighting the potential for self-improvement in LLMs.

% \section{Limitations}
% \label{sec:limitations}

\medskip
\newpage
\bibliography{neurips_2023}
\bibliographystyle{iclr2021_conference}

\newpage
\appendix
\section{Appendix}
\label{sec:appendix}
\clearpage

\renewcommand{\thefigure}{A.\arabic{figure}}
\renewcommand{\thetable}{A.\arabic{table}}
\renewcommand{\theequation}{A.\arabic{equation}}
\setcounter{figure}{0}
\setcounter{table}{0}
\setcounter{equation}{0}


\section{Qualitative Results}
\begin{figure*}[h!]
\centering
\includegraphics[width=\textwidth, keepaspectratio]{figs/qualitative_results_v3.pdf}%
\caption{\label{fig:qualitative} \textbf{Qualitative results of interacting with \agent} on unseen scenarios from a held-out test set of \agent-instruct. \agent's responses and actions can be grounded in novel scenes.}
  \vspace{-16pt}
\end{figure*}

\section{Data}\label{app:dataset}


\subsection{More Details on LEO-align}\label{app:dataset:leo_align}
\paragraph{Object-level caption.} To facilitate object-level grounding of detailed object attributes, we leverage Cap3D~\citep{luo2023scalable}, which contains language descriptions for objects in Objaverse~\citep{deitke2023objaverse}. Given a single 3D object as input, \agent will be asked to predict its caption.

\paragraph{Object-in-the-scene caption.} For a better understanding of how an object can be related to others (spatial relations, \etc) when situated in a 3D scene, we collect referring expressions of objects in scenes from existing datasets, including ScanScribe~\citep{zhu20233d} and ReferIt3D~\citep{achlioptas2020referit3d}. Further, we generate additional object-referring expressions on 3RScan~\citep{wald2019rio} scenes by prompting \ac{llm} (details in \cref{app:dataset:prompt}).
During alignment, \agent needs to predict these referring expressions given the object-centric 3D input of the scene and the referred object.

\paragraph{Scene-level caption.} Finally, we encourage \agent to capture scene-level descriptions of a 3D scene. These scene-level captions focus on global information depicting key objects in the scene as well as their attributes and functionalities, relations among multiple objects, and room types and styles. We leverage scene graph annotations~\citep{wald2019rio} and prompt \ac{llm} to produce a total of \textasciitilde{}20K captions. 
To further increase caption diversity, we propose a subgraph sampling strategy to prevent LLMs from always attending to certain notable facets of the scene (details in \cref{app:subgraph_sampling}). Similar to previous settings, \agent needs to predict these captions given the corresponding 3D input.

\subsection{More Details on LEO-instruct}\label{app:dataset:leo_instruct}

Below, we provide a comprehensive illustration of the data preparation process for these tasks and an overview of generated data in~\cref{fig:data_framework}. We list the corresponding instructions in \cref{sec:supp_leo_ds_examples}.

\paragraph{3D captioning.} The task is to produce a generic caption given 3D input. We adopt the Scan2Cap dataset~\citep{chen2021scan2cap}, which is based on the ScanNet~\citep{dai2017scannet} 3D scenes and covers various levels (object-level and scene-level) and aspects (attributes, relations, \etc) of scene details.

\paragraph{3D question answering.} The 3D-QA task is an extension of VQA~\citep{antol2015vqa} to 3D scenes with a focus on 3D knowledge, ranging from spatial relations to functionalities of objects. For this task, we first aggregate two existing 3D-QA datasets: ScanQA~\citep{azuma2022scanqa} and SQA3D~\citep{ma2023sqa3d}. To further generate questions concerning rich 3D knowledge, we prompt LLMs to generate \textasciitilde{}35K QA pairs on 3RScanQA with our quality refinement techniques discussed in~\cref{sec:data:generation}.



\paragraph{3D dialogue.} The goal of this task is to support natural conversations between \agent and users about a given 3D scene. 
This task necessitates coherence and continuity across multiple rounds of conversational interactions.
We build such dialogues on 3RScan scenes by prompting LLMs with a variant of the Chain-of-Thought prompting method discussed in~\cref{sec:data:generation} to facilitate diverse dialogues about relevant and accurate details about the 3D scene. 
In total, \textasciitilde{}11K dialogues are collected. 



\paragraph{Scene-aware task planning.} In this task, \agent is required to decompose high-level tasks into step-by-step low-level plans given 3D scenes. 
We expect \agent to generate feasible plans based on the current 3D scene and ground its inherent common sense knowledge about procedures to the scene configurations, including, objects, their attributes, relations, and functional characteristics, \etc. By prompting LLMs, we end up collecting \textasciitilde{}14K task-plan pairs on 3RScan scenes.


\paragraph{Embodied navigation.} We follow imitation learning setting in Habitat-web~\citep{ramrakhya2022habitat} for the embodied navigation task. We choose \texttt{ObjNav}, where \agent needs to map navigation instructions (\eg ``find bed''), object-centric 3D input, and an egocentric 2D input into discrete habitat motor commands. For simplicity, we use shortest path navigation trials rather than human demonstrations for learning as they are less noisy and therefore easier to learn when provided with the 3D scene. In total, we generate \textasciitilde{}60K navigation episodes out of the MP3D \texttt{ObjNav} training scenes~\citep{savva2019habitat} for this task.

\paragraph{Robotic manipulation.} We employ a subset of the manipulation tasks introduced in CLIPort~\citep{cliport}. The input of this task includes instructions, egocentric 2D observations, and object-centric 3D information. As discussed in~\cref{sec:model_tokenization}, we discretize the continuous action space of CLIPort into bins to unify the action decoding of navigation and manipulation (more details in~\cref{sec:action_tokenization}). We generate 100K demonstrations for each selected manipulation task. 

\subsection{Design of Seed Tasks for LLM-assisted 3D Data Generation}\label{app:dataset:seed_task}

\paragraph{Object Scene Caption \& Scene Caption.} To align the 3D scene/object with language, we prompt ChatGPT to curate these two types of caption data. Object Scene Caption includes the spatial relationships of the object with some adjacent objects in the scene. Scene Caption is the comprehensive description for the whole 3D scene, including some key objects and their spatial relationships.

\paragraph{QA \& Dialogue.} For QA, we design several question-answer pairs given a scene graph. A diverse set of questions are asked about the 3D scene, including the object attributes, object counting, object existence, spatial relationships between the objects, object types, object affordance, room type and so on. For dialogue, we design a conversation between the assistant and a person asking questions about this scene. The answers are in a tone as if the assistant is understanding the scene and helping the person. Different from single-round QA, dialogue has some high-level tasks such as 'searching for specific objects' that require multi-round conversations.

\paragraph{Planning.} To include a deeper understanding of the global 3D scene information, we prompt ChatGPT to generate a high-level task and 5-10 action steps(interaction between the assistant and the objects in the scene) to finish the task.

\subsection{Prompts for LLM-assisted 3D Data Generation}\label{app:dataset:prompt}
In \crefrange{fig:prompt:dialogue}{fig:prompt:object_caption}, we show the prompts for five types of LLM-assisted 3D-language data generation. We provide few-shot examples as the context. In each example, the ``content'' contains a scene graph, and the ``response'' refers to a human-labeled response. The query is a new scene graph, based on which ChatGPT \citep{openai2022chatgpt} generates responses.


\begin{figure}[t!]
\centering
\includegraphics[width=\textwidth, keepaspectratio]{figs/prompt_dialogue.pdf}%
  \caption{The prompt for generating 3D Dialogue.}
  \label{fig:prompt:dialogue}
\end{figure}

\begin{figure}[t!]
\centering
\includegraphics[width=\textwidth, keepaspectratio]{figs/prompt_QA.pdf}%
  \caption{The prompt for generating 3D QA.}
  \label{fig:prompt:QA}
\end{figure}

\cref{fig:prompt:dialogue} shows the prompt for generating 3D dialogue data. {\color{red}{Red fonts}} outline our requirements of the dialogue content, including object attributes, spatial relations, and commonsense topics. \textcolor{purple}{Purple fonts} formulate the template of the response. We require the response generated by the ChatGPT should include the dialogue context as well; the ``thought'' contains the involved objects in the question, which is used to enhance the reliability of the answer. These two components will be removed after the refinement procedures.

\begin{figure}[t!]
\centering
\includegraphics[width=\textwidth, keepaspectratio]{figs/prompt_planning.pdf}%
  \caption{The prompt for generating 3D planning.}
  \label{fig:prompt:planning}
\end{figure}

\begin{figure}[t!]
\centering
\includegraphics[width=\textwidth, keepaspectratio]{figs/prompt_scene_caption.pdf}%
  \caption{The prompt for generating 3D scene caption.}
  \label{fig:prompt:scene_caption}
\end{figure}

\begin{figure}[t!]
\centering
\includegraphics[width=\textwidth, keepaspectratio]{figs/prompt_object_caption.pdf}%
  \caption{The prompt for generating 3D object-in-the-scene caption. }
  \label{fig:prompt:object_caption}
\end{figure}

\subsection{Analysis of the Object-Centric Chain-of-Thought}\label{app:dataset:ocot}
\begin{table}[t!]
    \centering
    \small
    \caption{The effect of \ac{ocot} on the answer accuracy for Object Counting questions.}
    \begin{tabular}{l|c c c c c c }
        \toprule
        \textbf{Settings} & \textbf{Seed 1} & \textbf{Seed 2} & \textbf{Seed 3} & \textbf{Seed 4} & \textbf{Average} & \textbf{Avg. Gain} \\
        \midrule
        w/o \ac{ocot} & 0.5838 & 0.5349 & 0.5962 & 0.5816 & 0.5741 & \multirow{2}[2]{*}{0.2061}\\
        \cmidrule{1-6}
        O-CoT & 0.7647 & 0.8117 & 0.7778 & 0.7667 & 0.7802 \\
        \bottomrule
    \end{tabular}
    \label{tab:ocot_ablation}
\end{table}
To further investigate the impact of Object-centric Chain-of-Thought (\ac{ocot}) on data quality, we analyze the answer accuracy for Object Counting questions. Specifically, we collect several demonstrations, and for each run, we select two of them as the prompt seed. With these seeds, we generate dialogues across all scenes in 3DSSG \citep{wu2021scenegraphfusion} and then assess the answer accuracy for Object Counting questions. The results are presented in~\cref{tab:ocot_ablation}.


The results in \cref{tab:ocot_ablation} indicate that \ac{ocot} consistently improves the answer accuracy for Object Counting questions. Though there remain errors after applying \ac{ocot}, we will conduct refinement to fix them. Examples of Object Counting questions are provided in \cref{app:dataset:refine:examples}. 


\subsection{Refinement Details}\label{app:dataset:refine:examples}
We conduct refinement by passing raw LLM-generated responses into several human-defined filtering procedures based on the 3D scene graph. The refinement considers five raw response categories:
\begin{itemize}[leftmargin=*,nolistsep]
    \item Object Counting. The question concerns counting the target object.
    \item Object Existence. The response claims the existence of objects, which can be actually either existent or non-existent.
    \item Object Non-existence. The response claims the non-existence of objects, which can be actually either existent or non-existent.
    \item Negative Response. The scene graph cannot provide a solid response to the question, which means the question cannot be answered and will be discarded.
    \item Response with ID. The response contains unexpected object IDs.
\end{itemize} 

Specifically, we employ regular expression matching to detect errors in these five categories. We also employ this method to correct the responses except for Response with ID, which will be rewritten by ChatGPT instead. The QA pair will be eliminated if multiple rounds of rewriting fail to remove the IDs. \cref{tab:dialogue_refinement} and \cref{tab:qa_refinement} show some examples of the responses subject to the above five categories as well as the effect of our refinement. 
\begin{table}[t!]
    \centering
    \small
    \caption{\textbf{Examples of dialogue refinement}.}
    \begin{tabular}{l|p{4.5cm}|p{4.5cm}}
        \toprule
        \textbf{Types} & \textbf{Raw Responses} & \textbf{Refined Responses}\\
        \midrule
        \textbf{Object Counting} & There are 3 chairs in the room.
        
        I see there are two washing machines in the bathroom.
        & There are 4 chairs in the room.

        I see there are 4 washing machines in the bathroom. \\
        \midrule
        \textbf{Object Existence} & Yes, there is a cutting board in the kitchen.
        
        Yes, there is a computer and a monitor on the desk. However, the monitor is currently off.
        &  No, there is no cutting board in the room.
        
        No, there is no computer in the room.
        \\
        \midrule
        \textbf{Object Non-existence} & No, there is no stereo equipment in the room.
        
        I'm sorry, but I couldn't find a hair dryer in the bathroom.
        &Yes, there is a stereo equipment in the room.
        
        Yes, I found a hair dryer in the room.
        \\
        \midrule
        \textbf{Negative Response} &  
        No, there is nothing else mentioned in the scene graph. 
        
        I'm sorry, but there is no mention of a mirror in the scene graph for the bathroom.
        &

        \textit{The negative responses will be removed.} \\
        \midrule
        \textbf{Response with ID}  & You can place your backpack on the floor, to the left of the dining table-33. As for your bag, you can place it on the floor, to the left of the bed-10.
        &
        You can place your backpack on the floor, to the left of the dining table. As for your bag, you can place it on the floor, to the left of the bed.
        \\
        \bottomrule
    \end{tabular}
    \label{tab:dialogue_refinement}
\end{table}

\begin{table}[t!]
    \centering
    \small
    \caption{\textbf{Examples of QA refinement}.}
    \begin{tabular}{l|p{4.5cm}|p{4.5cm}}
        \toprule
        \textbf{Types} & \textbf{Raw Responses} & \textbf{Refined Responses}\\
        \midrule
        \textbf{Object Counting} & Q: How many chairs are in the room?

        A: 3
        & Q: How many chairs are in the room?

        A: four \\
        \midrule
        \textbf{Object Existence} & Q: Is there a mirror in the room?
        
        A: yes
        & Q: Is there a mirror in the room?
        
        A: no
        \\
        \midrule
        \textbf{Object Non-existence} & Q: Is there an ironing board in the room?
        
        A: no
        &Q: Is there an ironing board in the room?
        
        A: yes
        \\
        \midrule
        \textbf{Negative Response} &  
        Q: What is the material of the bathtub?
        A: unknown

        Q: Where is the shampoo dispenser?
        A: unknown
        &
        \textit{The negative responses will be removed.} \\
        \midrule
        \textbf{Response with ID}  & Q: Where is the mirror located?

        A: attached to wall-3, behind heater-18, to the left of shelf-19
        &
        Q: Where is the mirror located?

        A: attached to a wall, behind a heater, to the left of a shelf
        \\
        \bottomrule
    \end{tabular}
    \label{tab:qa_refinement}
\end{table}

\subsection{Subgraph Sampling}\label{app:subgraph_sampling}
To enhance the diversity of the 3D scene graphs used for prompting, we perform subgraph sampling on the 3DSSG according to a sampling rate, which denotes the ratio of preserved nodes. The sampled subgraphs are used for generating scene captions and planning data. We analyze the distribution of node numbers across the 3DSSG dataset in \cref{fig:node_distribution} and set different sampling rates for scenes with different numbers of nodes in \cref{tab:sampling_rate}. For each sampling rate, we set 4 random prompt seeds to further enhance the diversity of prompted data.
\begin{figure}[t!]
\centering
\includegraphics[width=0.6\textwidth,height=5.5cm]{figs/node_histogram.png}%
  \caption{\textbf{The distribution of node numbers for 3DSSG scenes.} The node number represents the number of objects in a scene.
  }
  \label{fig:node_distribution}
\end{figure}

\begin{table}[t!]
    \centering
    \small
    \caption{\textbf{Sampling rates for scenes with different node numbers.} The hyphen denotes a sweep of sampling rates, \eg, ``0.7-0.9'' means ``0.7,0.8,0.9''.}
    \begin{tabular}{l|c c c c c c c}
        \toprule
        \textbf{Node Number} & \textbf{10-20} & \textbf{20-30} & \textbf{30-40} & \textbf{40-50} & \textbf{50-60} &\textbf{60-70} &\textbf{>70} \\
        \midrule
        Sampling Rate & 0.8,0.9 & 0.7-0.9 & 0.6-0.9 & 0.6-0.9 & 0.5-0.9 & 0.5-0.9 & 0.4-0.9\\
        \bottomrule
    \end{tabular}
    \label{tab:sampling_rate}
\end{table}

To verify whether the subgraph sampling strategy can maintain the consistency and diversity of scene captions, we generate scene captions for the same scene using both the full graph and subgraph. We then employ GPT-4 \citep{openai2023gpt4} to evaluate the similarities and differences between the two captions.
The results in~\cref{tab:subgraph_analysis} indicate that our subgraph sampling strategy can maintain both consistency and diversity.

\begin{table}[t!]
    \centering
    \small
    \caption{\textbf{The comparative analysis of full graph prompting and subgraph prompting.}}
    \begin{tabular}{l|p{10cm}}
        \toprule
        \textbf{Settings} & \textbf{Generated Scene Captions}\\
        \midrule
        \textbf{Full Graph} & In this scene, there is a colorful and flat white wall. A small TV stand is positioned in front of a low upholstered couch, to the left of a brown lamp, and to the right of a wooden table. The couch is covered with a pink padded square pillow and a patterned blanket. A light hangs on the wall and another light is supported by the table. There is a rectangular glass window attached to the wall, behind which is a yellow curtain. The room has a comfortable and cozy atmosphere, suitable for relaxation and entertainment. \\
        \midrule
        \textbf{Sampling Rate 0.6} & In this scene, there is a colorful white wall attached to a brown wooden floor. The couch is upholstered and low in size, located to the right of a small TV stand and a table. A red blanket covers the couch, and there are patterned, padded square pillows on the couch. A black glass window is attached to the wall, and there are artificial lamps standing on the floor and close to the couch. The lamps are also near a table and to the right of the TV stand. The scene suggests a comfortable living room with a cozy couch, pillows, and a blanket, suitable for relaxation and entertainment. \\
        \midrule
        \textbf{GPT-4 Evalutaion} &  \textbf{Similarities:} 1. Both scenes describe a living room setting, emphasizing comfort and coziness suitable for relaxation and entertainment.
2. Both rooms have a white wall.
3. Each scene features a couch and a TV stand.
4. Both scenes describe a blanket and a padded square pillow on the couch.

\textbf{Differences:} 1. The first scene has a brown wooden floor, while the floor is not mentioned in the second scene.
2. The first scene has a red blanket on the couch; the second has a patterned blanket but doesn't specify the color.
3. The first scene describes the lamps as ``standing on the floor'', while the second mentions one light hanging on the wall and another supported by the table.
4. The second scene includes a yellow curtain behind the window, which the first scene does not mention.

\textbf{Summary:} Overall, both summaries provide a similar thematic view of a comfortable living room but differ in the specific arrangement and color details of the items within the room.
\\
        \bottomrule
    \end{tabular}
    \label{tab:subgraph_analysis}
\end{table}


\subsection{Scene-graph-based Prompting \vs Box-based Prompting}
\label{sec:scene graph prompting and bbox prompting}

\begin{figure}[t!]
\centering
\includegraphics[width=\textwidth, keepaspectratio]{figs/dialogue_compare_content.pdf}%
  \caption{Comparison of the content between box-based and scene-graph-based prompting.}
  \label{fig:prompt:compare_content}
\end{figure}

\begin{figure}[t!]
\centering
\includegraphics[width=\textwidth, keepaspectratio]{figs/dialogue_compare_demonstrations.pdf}%
  \caption{The demonstrations for prompting, which are designed to be similar for a fair comparison.}
  \label{fig:prompt:compare_demonstrations}
\end{figure}

\begin{figure}[!t]
\centering
\includegraphics[width=\textwidth, keepaspectratio]{figs/dialogue_compare_responses.pdf}%
  \caption{The responses of two prompting methods. Descriptions highlighted in {\color{red}{red}} show our method leads to more flexible and reliable spatial relations.}
  \label{fig:prompt:compare_responses}
\end{figure}

In this section, we provide a comparative analysis of scene-graph-based prompting and box-based prompting \citep{hong20233d}. We refer the readers to Figure 6 in 3D-LLM \citep{hong20233d} for details of the box-based prompting method. \cref{fig:prompt:compare_content} shows the contents of two methods. To present a fair comparison between the two methods, we prompt with 1) demonstrations that have similar content under the same scene (see \cref{fig:prompt:compare_demonstrations}) and 2) identical new scene queries. Since 3D-LLM does not elaborate on attribute-related prompts, we mainly compare the spatial relations in the responses. As shown in \cref{fig:prompt:compare_responses}, we highlight some spatial relations in {\color{red}{red}}. The comparison shows that our method provides more diverse and reliable spatial relations, which are important for 3D scene understanding.

\subsection{Additional Comparision Regarding Dataset Quality}
\label{app:additional_data_comparison}
In addition to assessing the factual accuracy of responses compared to 3D-LLM, we also compared the grammatical correctness of the responses with ScanScribe\cite{zhu20233d}, a template-based synthetic dataset that focuses on 3D object caption. We observed that their dataset exhibited some grammar errors, whereas our dataset did not manifest such issues. We provide some data examples in \cref{tab:grammar_compare_1} and \cref{tab:grammar_compare_2}. We highlighted the grammar errors present in ScanScribe dataset in {\color{red}{red}}. Through comparison, it is evident that our sentences exhibit accurate and natural syntax, and also surpasses ScanScribe in the diversity and complexity of object descriptions.

\begin{table}[htbp]
\centering
\caption{Object captions in the 3Rscan scene 8f0f144b-55de-28ce-8053-2828b87a0cc9.}
\label{tab:grammar_compare_1}
\begin{tabular}{@{}l|l|l|p{10cm}}
\toprule
object label-id & method     & response id & caption                                                                                      \\ \midrule
microwave-8     & ours       & 1           & The microwave is black in color and is standing on the table. It is located behind and to the left of the toaster. \\ \cmidrule(lr){3-4}
                &            & 2           & The microwave is black in color. It is standing on a table and is positioned behind and to the left of a toaster.   \\ \cmidrule(lr){2-4}
                & ScanScribe & 1           & This is a black microwave. It is standing on the table. It is close by the toaster. A black microwave is in front of {\color{red}{the this}} microwave. \\ \cmidrule(lr){3-4}
                &            & 2           & This is a black microwave. It is behind the toaster. It is standing on the table. A black microwave is in front of {\color{red}{the this}} microwave.   \\ \bottomrule
\end{tabular}
\end{table}


\begin{table}[htbp]
\centering
\caption{Object captions in the 3Rscan scene 0cac753c-8d6f-2d13-8e27-e0664fc33bb9.}
\label{tab:grammar_compare_2}
\begin{tabular}{@{}l|l|l|p{10cm}}
\toprule
object label-id & method     & response id & caption                                                                                      \\ \midrule
counter-4    & ours       & 1           & The counter is standing on a wooden floor. It is close to 4 stools and an armchair. The stools are simple and wide in size, while the armchair is narrow in size. The floor is wooden and flat, and it is clean. \\ \cmidrule(lr){3-4}
                &            & 2           & The counter is standing on the wooden floor. It is close to 4 stools and an armchair. The floor is made of wooden material and is flat in shape. It is also clean. The armchair is narrow in size while the stools are simple in design. Three of the stools are wide in size.   \\ \cmidrule(lr){2-4}
                & ScanScribe & 1           & This is a counter. It is close by the armchair. It is close by the stool. A counter is close by the this counter. A counter is close by {\color{red}{the this}} counter.
                         \\ \bottomrule
\end{tabular}
\end{table}


\subsection{Dataset Statistics}
\label{app:dataset statistics}

\begin{figure}[t!]
    \centering
    \begin{minipage}[b]{0.45\textwidth}
        \includegraphics[width=1\textwidth]{figs/Rscan_qa_question_stat.png}
        \caption{Question types: 3RQA.}
        \label{fig:RscanQA_statistics}
    \end{minipage}
    \hfill
    \begin{minipage}[b]{0.467\textwidth}
        \includegraphics[width=1\textwidth]{figs/Rscan_dialogue_question_stat.png}
        \caption{Question types: 3RDialog.}
        \label{fig:RscanDialog_Q_statistics}
    \end{minipage}
\end{figure}

\begin{figure}[t!]
    \centering
    \begin{minipage}[b]{0.45\textwidth}
        \includegraphics[width=1\textwidth]{figs/Rscan_dialogue_noun_verb_instruction_stat.png}
        \caption{Noun-verb pairs: 3RDialog instruction.}
        \label{fig:RscanDialog_noun_verb_instruction}
    \end{minipage}
    \hfill
    \begin{minipage}[b]{0.46\textwidth}
        \includegraphics[width=1\textwidth]{figs/Rscan_plan_noun_verb_instruction_stat.png}
        \caption{Noun-verb pairs: 3RPlan instruction.}
        \label{fig:RscanPlan_noun_verb_instruction}
    \end{minipage}
\end{figure}   

\begin{figure}[t!]
    \centering
    \begin{minipage}[b]{0.45\textwidth}
        \includegraphics[width=1\textwidth]{figs/Rscan_dialogue_noun_verb_response_stat.png}
        \caption{Noun-verb pairs: 3RDialog response.}
        \label{fig:RscanDialog_noun_verb_response}
    \end{minipage}
    \hfill
    \begin{minipage}[b]{0.433\textwidth}
        \includegraphics[width=1\textwidth]{figs/Rscan_plan_noun_verb_response_stat.png}
        \caption{Noun-verb pairs: 3RPlan response.}
        \label{fig:RscanPlan_noun_verb_response}
    \end{minipage}
\end{figure}   

We provide statistics on the instruction-tuning datasets. We visualize the distribution of the question types in 3RQA (\cref{fig:RscanQA_statistics}) and 3RDialog (\cref{fig:RscanDialog_Q_statistics}). The pie chart's inner circle represents the first word of the questions, while the outer circle accounts for the second or third word in the corresponding questions. The results show that the questions cover the attributes and spatial relations of the objects, as well as high-level topics such as room types and functionalities.

We also provide statistics of the root noun-verb pairs for instructions and responses in 3RDialog and 3RPlan, as shown in \crefrange{fig:RscanDialog_noun_verb_instruction}{fig:RscanPlan_noun_verb_response}.

\section{Data Examples}\label{sec:supp_leo_ds_examples}
Please refer to \crefrange{tab:supp_data_example}{tab:supp_data_example_cont2} for examples of our dataset.

\section{Model Details}\label{app:model}
\subsection{Prompts}
The first portion of prompts sent into the LLM is a \textbf{system message}. It consists of two parts: a role prompt and a situation prompt. The role prompt is the same for all tasks:
\begin{tcolorbox}
\begin{minipage}{\linewidth}
You are an AI visual assistant situated in a 3D scene. You can perceive (1) an ego-view image (accessible when necessary) and (2) the objects (including yourself) in the scene (always accessible). You should properly respond to the USER's instructions according to the given visual information.
\end{minipage}
\end{tcolorbox}

The situation prompt begins with a common sentence:

\begin{tcolorbox}
\begin{minipage}{\linewidth}
You are at a selected location in the 3D scene.
\end{minipage}
\end{tcolorbox}

For SQA3D~\citep{ma2023sqa3d}, the situation prompt is further extended with the situation description in the dataset. The situation prompt is only used jointly with the embodiment token to support tasks that require information about the embodiment. Details can be found in \cref{sec:supp_embodiment}.

Next are the \textbf{visual tokens}, including \textbf{2D image tokens} and \textbf{object-centric 3D tokens}. Each token sequence is interleaved within text tokens and starts with a text prefix.
\begin{tcolorbox}
\begin{minipage}{\linewidth}
Ego-view image: \{\texttt{IMAGE\_TOKENS}\} \\
Objects (including you) in the scene: \{\texttt{OBJECT\_TOKENS}\}
\end{minipage}
\end{tcolorbox}

The last portion of prompts is a \textbf{task-specific instruction}. For \textbf{object-level caption} and \textbf{object-in-the-scene caption}, we randomly chose one sentence from 151 sentences to be the instruction. Some examples can be found in \cref{tab:obj-level_cap}. For \textbf{scene-level caption}, we randomly choose one from 183 instructions. Examples can be found in \cref{tab:scene-level_cap}. For \textbf{3D question answering} task, we simply use the question as the instruction. The dialog history is used as the instruction for \textbf{3D dialogue} to provide continuity across multiple rounds of interactions. A planning instruction pool consisting of 202 instructions is introduced for \textbf{scene-aware task planning} and we randomly choose one from it as done in the caption tasks. Examples from the pool can be found in \cref{plan cap}. The chosen instruction is further followed by an instruction that specifies the task, \eg, \textit{set up a home office}.

With past action tokens \{\texttt{PAST\_ACTIONS}\} appended at the end, the instruction for \textbf{embodied navigation} is as follows, where \{\texttt{GOAL}\} stands for the goal specified by the target object name:
\begin{tcolorbox}
\begin{minipage}{\linewidth}
The task is navigation. Your goal is to find \{\texttt{GOAL}\} by moving around in the scene. Past actions: \{\texttt{PAST\_ACTIONS}\}.
\end{minipage}
\end{tcolorbox}

The instruction for \textbf{robotic manipulation} is similar to the one in \textbf{embodied navigation}. Here \{\texttt{GOAL}\} is the task description in CLIPort:
\begin{tcolorbox}
\begin{minipage}{\linewidth}
The task is manipulation. Your goal is to \{\texttt{GOAL}\}. Past actions: \{\texttt{PAST\_ACTIONS}\}.
\end{minipage}
\end{tcolorbox}


\begin{table}[t!]
\caption{\textbf{Examples from our object-level caption instruction set.}}
\vspace{-4pt}
\begin{tcolorbox}
\begin{minipage}{\linewidth}
"Produce a description for the object at the chosen spot in the 3D scene.",\\
    "How would you depict the object located at the selected point in the 3D environment?",\\
    "Formulate a description of the item at the picked position within the 3D scene.",\\
    "How would you describe the entity at the designated location in the 3D backdrop?",\\
    "Can you detail the object situated at the selected point in the 3D setting?",\\
    "Compose a narrative for the object at the chosen locale within the 3D environment.",\\
    "What does the object at the specified position in the 3D visualization look like?",\\
    "Provide a description for the item located at the marked site in the 3D world.",\\
    "How would you illustrate the object placed at the selected spot in the 3D landscape?",\\
    "Craft a depiction of the object at the pinpointed location within the 3D territory.",\\
    "What kind of object is illustrated at the identified site in the 3D tableau?",\\
    "Develop a description of the object at the specified position in the 3D backdrop.",\\
    "What is the entity's detail at the highlighted site in the 3D view?",\\
    "Write up a description of the entity at the selected spot in the 3D realm.",\\
    "What does the object look like at the pinpointed location in the 3D space?",\\
    "Detail the entity located at the chosen position within the 3D scene.",\\
    "Can you explain the essence of the object at the selected spot in the 3D zone?",
\end{minipage}
\end{tcolorbox}

\label{tab:obj-level_cap}
\end{table}
\begin{table}[t!]
\caption{\textbf{Examples from our scene-level caption instruction set.}}
\vspace{-4pt}
\begin{tcolorbox}
\begin{minipage}{\linewidth}
    "Describe this scene.",\\
    "Generate a description of this scene.",\\
    "Generate a caption of this scene.",\\
    "Can you describe the scene?",\\
    "Can you generate a description of the scene?",\\
    "Can you generate a caption of the scene?",\\
    "Summarize this scene.",\\
    "Provide an outline of this 3D scene's characteristics.",\\
    "How would you describe the 3D scene?",\\
    "How would you summarize this scene?",\\
    "Convey a summary of the 3D structure of this scene.",\\
    "How would you interpret this 3D scene?",\\
    "Offer a summary of the 3D scene.",\\
    "Can you describe this scene in detail?",\\
    "I'm interested in this scene, can you explain?",\\
    "What is this scene made of?",\\
    "Could you provide more info about this scene?",
\end{minipage}
\end{tcolorbox}

\label{tab:scene-level_cap}
\end{table}

\begin{table}[t!]
\caption{\textbf{Examples from our planning instruction pool.}}
\vspace{-4pt}
\begin{tcolorbox}
\begin{minipage}{\linewidth}
    "Plan for the task",\\
    "Can you come up with a plan for this task",\\
    "How can we do this task, provide a step-by-step plan",\\
    "Draft a plan for completing this task",\\
    "Detail a strategy for the task",\\
    "What's the best plan for this task",\\
    "Draw out a procedure for the task",\\
    "Lay out the steps for this task",\\
    "Could you devise a plan for the task",\\
    "Show me a plan for this task",\\
    "I need a plan for the task",\\
    "Sketch a plan for the task at hand",\\
    "Set up a plan for this",\\
    "Recommend a plan for this task",\\
    "Offer a strategy for this task",\\
    "Design a blueprint for the task",\\
    "Outline the approach for this task",
\end{minipage}
\end{tcolorbox}

\label{plan cap}
\end{table}

\subsection{Feature Encoding}\label{sec:supp_embedding}
We have several modules to encode the multi-modal features.
\begin{itemize}[leftmargin=*]
    \item \textbf{Object-centric 3D token embedding.} The encoder for 3D object-centric point clouds is a PointNet++~\citep{qi2017pointnet++} pre-trained on ScanNet~\citep{dai2017scannet} with object-classfication task. We sample 1024 points for every object as in~\cite{chen2022language}. The architecture parameters all remain the same with~\cite{chen2022language}. We freeze the PointNet++ for empirically better results. %

    \item \textbf{Spatial Transformer~\citep{chen2022language}.} Spatial Transformer is a modified transformer architecture that explicitly encodes spatial relations between object pairs. Specifically, consider the vanilla self-attention~\citep{vaswani2017attention} mechanism which takes as input a feature matrix $X\in \mathbf{R}^{N\times d}$, where $N$ stands for the number of tokens and $d$ is the feature dimension. Vanilla self-attention first compute $Q=XW_Q, K=XW_K, V=XW_V$ from $X$ using learnable projection matrices $W_Q, W_K, W_V\in \mathbf{R}^{d\times d_h}$ where $d_h$ stands for the output feature dimension. Then the attention weight matrix is computed by $(\omega^o_{ij})_{N\times N} = \Omega^o = softmax(\frac{QK^T}{\sqrt{d_h}})$ and finally used for re-weighting $\Omega^oV$. The intuition of Spatial Transformer is that we can re-scale the elements $\omega_{ij}^o$ in the weight matrix $\Omega^o$.
    
    In the object-centric reasoning setting, the input feature matrix is $O\in \mathbf{R}^{N\times d}$. Consider an object pair $(O_i, O_j)$ with their geometric centers $c_i, c_j$. Spatial Transformer~\citep{chen2022language} computes the Euclidean distance $d_{ij} = ||c_i-c_j||_2$ and the horizontal and vertical angles $\theta_h, \theta_v$ of the line connecting $c_i$ and $c_j$. The spatial feature between the two objects $(O_i, O_j)$ is a 5-dimensional vector $f_{ij} = [d_{ij}, \sin{(\theta_h)}, \cos{(\theta_h)}, \sin{(\theta_v)}, \cos{(\theta_v)}]$. To combine this feature with objects, the spatial attention computes $\omega^s_{ij} = g_i f_{ij}$ where $g_i=W_S^To_i$ is a 5-dimensional vector. The spatial attention further reweights the original self-attention weight matrix as
    $$
    \omega_{ij}=\frac{\sigma(\omega^s_{ij})exp(\omega^o_{ij})}{\sum_{l=1}^N\sigma(\omega^s_{il})exp(\omega^o_{il})}.
    $$
    Readers are referred to \cite{chen2022language} for more details. In summary, Spatial Transformer explicitly computes pairwise spatial relations and fuses them with vanilla self-attention to provide better spatial reasoning ability. We use a three-layer Spatial Transformer with 8 heads to process the object-centric features produced by PointNet++ and output object tokens for LLM. For other settings, We follow all the default hyperparameters in \cite{chen2022language}.
    
    \item \textbf{2D token embedding.} We use OpenCLIP ConvNext-base model~\citep{liu2022convnet} pre-trained on LAION2B~\citep{schuhmann2022laion} to process the egocentric 2D image.
    \item \textbf{CLIP semantic guidance.} To inject more semantics into visual tokens, we use the text encoder from CLIP~\citep{radford2021learning} to process the instruction tokens to obtain a global semantics feature. Next, we update the visual tokens via element-wise product between the CLIP semantics feature and each image \& object token embedding.
\end{itemize}

\subsubsection{Embodiment Encoding}\label{sec:supp_embodiment}
In addition to the egocentric 2D input, we introduce an embodiment token to help \agent reason in an embodiment-aware fashion. We find it useful to use it together with the situation prompt and 2D egocentric input. Specifically, an embodiment token $e$ is introduced in \textbf{embodied navigation}, \textbf{embodied reasoning}, and \textbf{object-in-the-scene caption} tasks. Specifically, $e$ is a learnable embedding that will be inserted into the 3D object list.

So what does embodiment information mean in these tasks? In \textbf{embodied navigation}, it means the agent's position and orientation in the scene, which can be derived from a GPS and a compass sensor. The orientation of the agent is further represented by a rotation which is Fourier-embedded and mapped to a feature vector $r$ by a linear layer. It is the same in \textbf{embodied reasoning} task.
In the \textbf{object-in-the-scene caption} task, we assume the agent is situated at the location of the object that is being referred to. Therefore, embodiment information also means the location of the referred object. We obtain this location by randomly choosing a spot inside the referred object bounding box. To sum up, we could simply treat the embodiment token as a special \textit{self object}, where its object embedding is learnable, and its location/orientation corresponds to the actual or assumed ``agent''.

After inserting the embodiment token, we obtain a new 3D object token list: $e, s_{\text{3D}}^{(1)}, s_{\text{3D}}^{(2)}, \dots, s_{\text{3D}}^{(N)}$, where $s_{\text{3D}}^{(i)}, i\in \{1, 2, \dots, N\}$ are 3D object token embeddings produced by PointNet++, along with location specified for each object (including the \textit{self-object}). We can concatenate them together to get a feature matrix $O\in \mathbf{R}^{(N+1)\times d}$ and send them to the Spatial Transformer to explicitly fuse the spatial information of all the 3D objects and the self-object.

\subsection{Action Tokenization}\label{sec:action_tokenization}

To empower \agent to exert control over an embodiment or a robot, we encode all actions within the context of Object Navigation~\citep{ramrakhya2022habitat} and CLIPort~\citep{cliport} tasks using the least frequently employed language tokens. Specifically, for the Object Navigation task, we allocate 4 tokens to represent actions of \textit{move forward}, \textit{turn right}, \textit{turn left}, and \textit{stop}. For the CLIPort task, we use a total of 516 tokens to discretize action poses, with 320 tokens dedicated to the x-axis pose bins, 160 tokens for the y-axis pose bins, and 36 tokens for the z-rotation bins.

\subsection{LLM Hyperparameters}
We set the maximum output length of our Vicuna-7B to be 256. The maximum context length is also set to 256 and if the length of the input is greater than 256, we truncate it to 256 by deleting tokens from the left (\ie, only the rightmost 256 tokens are preserved). We set rank and $\alpha$ in LoRA~\citep{hu2022lora} to be 16 and the dropout rate to be 0. LoRA is implemented for all the projection matrices in the LLM, \ie, $(W_q, W_k, W_v, W_o)$ in attention modules and $(W_{gate}, W_{up}, W_{down})$ in MLPs.

The hyperparameters for inference are listed in \cref{tab:parameter_beam}.


\section{Alignment Setup}
The hyperparameters for 3D VL alignment are presented in \cref{tab:param_align}.

\begin{table*}[t!]
\centering
\small
\caption{Hyperparameters for \agent inference.}\label{tab:parameter_beam}
\begin{tabular}{@{}ll@{}}
\toprule
\textbf{Hyperparameters}  & \textbf{Value}   \\ \midrule
Number of beams          & 5     \\
Maximum output length    & 256   \\
Minimum output length    & 1     \\
Top $p$                   & 0.9   \\ 
Repetition penalty       & 3.0   \\
Length penalty           & 1.0   \\
Temperature              & 1.0   \\
\bottomrule
\end{tabular}
\end{table*}

\begin{table*}[t!]
\centering
\small
\caption{Hyperparameters for the alignment stage.} \label{tab:param_align}
\begin{tabular}{@{}ll@{}}
\toprule
\textbf{Hyperparameter}  & \textbf{Value}   \\ \midrule
Optimizer                & AdamW     \\
Weight decay             & 0.05     \\
Betas                    & [0.9, 0.999]      \\
Learning rate            & $3\times 10^{-4}$ \\
Warmup steps             & 400      \\
Number of workers        & 4         \\
Parallel strategy        & DDP       \\
Type of GPUs             & NVIDIA A100       \\
Number of GPUs           & 4 \\
Accumulate gradient batches & 5      \\
Batch size per GPU (total)   & 4 (80)    \\ 
Training precision       & bfloat16 \\
Gradient norm            & 5.0        \\
Epochs                   & 5        \\
\bottomrule
\end{tabular}
\end{table*}

\section{Instruction-tuning Setup}
The hyperparameters for 3D VLA instruction tuning are presented in \cref{tab:param_sft}.
\begin{table*}[t!]
\centering
\small
\caption{Hyperparameters for the instruction-tuning stage.} \label{tab:param_sft}
\begin{tabular}{@{}ll@{}}
\toprule
\textbf{Hyperparameter}  & \textbf{Value}   \\ \midrule
Optimizer                & AdamW     \\
Weight decay             & 0.05     \\
Betas                    & [0.9, 0.999]      \\
Learning rate            & $3\times 10^{-5}$ \\
Warmup steps             & 400      \\
Number of workers        & 4         \\
Parallel strategy        & DDP       \\
Type of GPUs             & NVIDIA A100       \\
Number of GPUs           & 4 \\
Accumulate gradient batches & 5      \\
Batch size per GPU (total)   & 4 (80)    \\ 
Training precision       & bfloat16      \\
Gradient norm            & 5.0        \\
Epochs                   &  10        \\
\bottomrule
\end{tabular}
\end{table*}

\section{Ablation Details}\label{sec:supp_ablation}



\subsection{Object-centric Mask}
\paragraph{Ground truth \vs object proposals.} As we adopt an object-centric 3D representation, the object-centric masks are necessary to segment the scene point cloud. For scenes that lack annotations of object-centric masks, we can utilize off-the-shelf detection or segmentation models to generate object proposals and thus obtain the masks. We compare the performances of \agent (\textit{w/o Act}) between using ground-truth masks and Mask3D \citep{schult2022mask3d} proposals. The results in \cref{tab:mask3d_gap} indicate that using Mask3D proposals leads to a moderate performance drop on Scan2Cap (mainly due to the IoU@0.5 metrics) and comparable performances on QA tasks.

\begin{table}[t!]
\centering
\captionof{table}{Quantitative comparison between \agent (\textit{w/o Act}) using ground-truth masks and Mask3D proposals. Metrics follow \cref{tab:vl_results}.}
\resizebox{0.83\linewidth}{!}{
\begin{tabular}{lccccccccccc}
    \toprule
     & \multicolumn{5}{c}{Scan2Cap (val)} & \multicolumn{5}{c}{ScanQA (val)} & SQA3D (test) \\
     \cmidrule(lr){2-6} \cmidrule(lr){7-11} \cmidrule(lr){12-12}
     & C & B-4 & M & R & Sim & C & B-4 & M & R & EM@1 & EM@1 \\
    \midrule

    \textit{w/o Act} (Mask3D) & 72.4 & 38.2 & 27.9 & 58.1 & 55.3 & 101.4 & 13.2 & 20.0 & 49.2 & \textbf{24.5} \textcolor{gray}{(47.6)} & 50.0 \textcolor{gray}{(52.4)} \\

    \textit{w/o Act} (GT) & \textbf{87.4} & \textbf{44.5} & \textbf{30.8} & \textbf{65.7} & \textbf{65.4} & \textbf{103.0} & \textbf{14.6} & \textbf{20.1} & \textbf{49.7} & 24.3 \textbf{\textcolor{gray}{(48.5)}} & 50.0 \textbf{\textcolor{gray}{(52.5)}} \\
    \bottomrule
\end{tabular}
}
\label{tab:mask3d_gap}
\end{table}

\subsection{Model Ablation}\label{sec:model_ablation}
\paragraph{LLM.} Following the setting of \agent (\textit{w/o Act}), we ablate the default LLM (Vicuna-7B) with OPT-1.3B \citep{zhang2022opt} and Vicuna-13B \citep{vicuna2023}, respectively. We report the evaluation results on ScanNet and 3RScan tasks in \cref{tab:llm_ablation}. The results show a significant gap between OPT-1.3B and Vicuna-7B and comparable performances between Vicuna-7B and Vicuna-13B. This indicates the notable improvements when scaling from smaller LLM to 7B scale and the potential saturation if we continue to scale up, resembling the finding in \cref{sec:exp_scaling}.

\paragraph{Point cloud backbone.} We have tried substituting PointNet++ \citep{qi2017pointnet++} with Point-BERT \citep{yu2022point} as the point cloud backbone. Specifically, we utilize the Point-BERT checkpoint from PointLLM \citep{xu2023pointllm}, which has adapted Point-BERT to 6-channel (XYZRGB) input and learned a language-aligned representation for 3D objects. We have not observed notable difference between the performances of using Point-BERT and PointNet++ so we omit the results here.

\begin{table}[t!]
    \centering
    \captionof{table}{Quantitative results of \agent equipped with LLMs at different scales. Metrics follow \cref{tab:data_ablation}.}
    \setlength\tabcolsep{3pt}
    \resizebox{0.65\linewidth}{!}{
    \begin{tabular}{lcccccc}
        \toprule
     & \multicolumn{3}{c}{ScanNet} & \multicolumn{3}{c}{3RScan} \\
     \cmidrule(lr){2-4} \cmidrule(lr){5-7}
         & Scan2Cap & ScanQA & SQA3D & 3RQA & 3RDialog & 3RPlan \\
        \midrule
        \textit{w/o Act} (OPT-1.3B) & 64.6 & 20.3 \textcolor{gray}{(44.2)} & 45.5 \textcolor{gray}{(47.6)} & 50.0 \textcolor{gray}{(54.5)} & 71.1 & 78.3 \\
        \textit{w/o Act} (Vicuna-7B) & \textbf{65.4} & \textbf{24.3} \textcolor{gray}{(48.5)} & \textbf{50.0 \textcolor{gray}{(52.5)}} & 51.9 \textcolor{gray}{(57.4)} & \textbf{73.3} & \textbf{81.1} \\
        \textit{w/o Act} (Vicuna-13B) & 65.2 & 23.4 \textbf{\textcolor{gray}{(48.9)}} & 49.7 \textcolor{gray}{(52.3)} & \textbf{56.2 \textcolor{gray}{(60.4)}} & 72.5 & 80.5 \\
        \bottomrule
    \end{tabular}
    }
    \label{tab:llm_ablation}
\end{table}

\subsection{Dialogue and Planning Data}\label{sec:dialog_planning}
To evaluate \textit{w/o Dialg}, we design an evaluation set with three types of questions: 1) \textbf{Answerable}: general questions that can be answered based on the given 3D scenes; 2) \textbf{Unanswerable}: questions that cannot be answered given the 3D scenes due to a lack of information, \eg, ``Tell me about the elephant in the room''; 3) \textbf{NLP}: questions that solely examine the language functionality of \agent in term of factual knowledge, reasoning, and text coherence. We collect 30 representative questions for each subset and generate \agent's responses for each question. We then ask humans to choose their preferred responses between \textit{w/o Dialg} and \textit{w/ Dialg} Based on the human preferences, we evaluate the two models with TrueSkill~\cite{graepel2007bayesian}, which is an algorithm that quantifies players’ rating scores by Bayesian inference. The scores are estimated by Gaussian distribution and expressed as $\mu\pm\sigma$.

\subsection{Data Balancing}\label{sec:data_balancing}
To investigate the hallucination problem, we collect 150 questions querying object existence on 3RScan and ScanNet respectively. We split three subsets according to the category of queried object. The queried object can exist in the given scene (Yes),
exist in other scenes instead of the given scene (No-1), or not exist in all the scenes (No-2).
Each subset comprises 50 questions. We merge No-1 and No-2 when reporting the exact-match accuracy, as shown in \cref{tab:data_balance}.

\section{Evaluation Details}

\subsection{3D Question Answering}\label{sec:supp_eval_qa}
\paragraph{Rationality of QA evaluation protocol.} We argue that exact match (EM), as a conventional metric for 3D QA, is unsuitable for evaluating the open-ended answer generated by LLMs. For example, given the question ``\textit{On what side of the towel is a bathroom curtain}?'' with ground-truth answer ``\textit{left side of towel}'', it is never wrong to answer ``left''. However, this will be deemed incorrect if we adopt the strict exact match protocol. Such a misjudgment is quite likely to occur when evaluating the answers from LLMs. By contrast, the classifier heads for QA (\eg, MCAN) are less affected because they collect all possible answers in advance to formulate the QA as a close-set classification problem. Hence, we refine the strict exact match protocol as follows.

\begin{lstlisting}[language=Python]
"""
code for QA protocols
pred: str
gts: List[str]
"""

def strict_em(pred, gts):
    for gt in gts:
        if pred == gt:
            # case 1
            return True


def refined_em(pred, gts):
    for gt in gts:
        if pred == gt:
            # case 1
            return True
        elif ''.join(pred.split()) in ''.join(gt.split()):
            # case 2
            return True
        elif ''.join(gt.split()) in ''.join(pred.split()):
            # case 3
            return True
    return False
\end{lstlisting}

In a nutshell, we squeeze the \texttt{pred} and \texttt{gt}, and then check whether one is a subset of the other. To justify our refined exact match protocol, in \cref{tab:em_protocol_cases} we provide some representative examples in the ScanQA validation set. Despite the improvements, we speculate such a simple refinement is still insufficient for a sound evaluation metric considering the flexibility of human language.
\begin{table}[t!]
    \centering
    \caption{Examples from ScanQA validation set manifest the rationality of our refined exact match protocol.}
    \resizebox{\linewidth}{!}{
    \begin{tabular}{lllcc}
        \toprule
        Question & Ground-truth answer & Predicted answer & Strict EM & Refined EM \\
        \midrule
        What color is the chair in the kitchen? & dark brown & brown & \xmark & \cmark (case 2) \\
        What is under the long kitchen counter? & kitchen cabinets & brown rectangular kitchen cabinets & \xmark & \cmark (case 2) \\
        What type of refrigerator is on the right of a kitchen counter? & stainless steel refrigerator & stainless steel & \xmark & \cmark (case 2) \\
        Where is the beige wooden desk placed? & up against wall & against wall & \xmark & \cmark (case 2) \\
        What color does the sofa look? & it looks black & black & \xmark & \cmark (case 2) \\
        Where is the black office chair located? & in front of desks & in front of desk & \xmark & \cmark (case 2) \\
        What is in the corner by windows? & book shelf & bookshelf & \xmark & \cmark (case 2) \\
        Where is the chair pulled into? & table & under table & \xmark & \cmark (case 3) \\
        How many chairs are to the left of the table? & 4 & 4 chairs & \xmark & \cmark (case 3) \\
        What objects are sitting on the black couch? & pillow & pillows & \xmark & \cmark (case 3) \\
        Where are the two different size tables located in room? & in center & in center of room & \xmark & \cmark (case 3) \\
        Where is the laptop located? & desk & on desk & \xmark & \cmark (case 3) \\
        Where is the soap dispenser mounted & above sink & on wall above sink & \xmark & \cmark (case 3) \\
        \bottomrule
    \end{tabular}
    }
    \label{tab:em_protocol_cases}
\end{table}

\subsection{Embodied Navigation}\label{sec:supp_eai_split}
To construct our training set, we adopt all 57 scenes in the MP3D \texttt{ObjNav} training split~\citep{savva2019habitat,ramrakhya2022habitat} and generate \textasciitilde{}60K shortest-path navigation episodes. The evaluation is conducted on the original validation split of the MP3D \texttt{ObjNav} task and the newly introduced HM3D \texttt{ObjNav} task~\citep{ramakrishnan2021habitat}. 

In contrast to most \texttt{ObjNav} agents that utilize recurrence through either RNN~\citep{ramrakhya2022habitat} or DT-style Transformer~\citep{suglia2021embodied}, \agent only employs a simplistic feed-forward policy, \ie, the Transformer in \agent only takes in the instruction, current state (2D and 3D observation), and past 4 actions, and predicts the next action, similar to RT-2~\citep{brohan2023rt}. Therefore, the only information relayed from the past is past actions. The absence of recurrence in \agent's acting policy is indeed the result of a trade-off between better performances and training efficiency. We will commit to exploring the possibility of looping in more sophisticated policy architectures (\eg, recurrence) in future work.



 

\section{Additional Results}\label{sec:additional_results}

\subsection{Impact of Data Refinement}\label{sec:impact_data_refinement}
\paragraph{Settings.} We investigate the impact of data refinement by comparing the downstream performances between pretraining on the generated data before/after refinement. Specifically, since our generated data (where the refinement occurs) pertains to 3RScan scenes, we first pretrain the \agent after the alignment stage on a mix of 3RScan datasets, and then train on a mix of ScanNet datasets (Scan2Cap, ScanQA, and SQA), where we report the quantitative results as downstream performances.

\begin{table}[t!]
\centering
\captionof{table}{Quantitative comparison between \agent pretrained on the generated data before/after refinement. Metrics follow \cref{tab:vl_results}.}
\resizebox{0.83\linewidth}{!}{
\begin{tabular}{lccccccccccc}
    \toprule
     & \multicolumn{5}{c}{Scan2Cap (val)} & \multicolumn{5}{c}{ScanQA (val)} & SQA3D (test) \\
     \cmidrule(lr){2-6} \cmidrule(lr){7-11} \cmidrule(lr){12-12}
     & C & B-4 & M & R & Sim & C & B-4 & M & R & EM@1 & EM@1 \\
    \midrule

    Before refinement & 84.1 & \textbf{45.8} & 30.9 & 66.1 & 65.3 & 99.4 & 12.6 & 19.4 & 48.6 & 24.5 \textcolor{gray}{(49.1)} & 48.2 \textcolor{gray}{(50.5)} \\

    After refinement & \textbf{87.1} & 45.2 & \textbf{31.1} & 66.1 & \textbf{65.7} & \textbf{105.7} & \textbf{14.9} & \textbf{20.5} & \textbf{50.7} & \textbf{24.7 \textcolor{gray}{(49.8)}} & \textbf{52.4 \textcolor{gray}{(55.0)}} \\
    \bottomrule
\end{tabular}
}
\label{tab:impact_data_refinement}
\end{table}

The results in \cref{tab:impact_data_refinement} demonstrate that data refinement elicits consistent improvements. In particular, data refinement primarily benefits reasoning (QA) tasks, probably because the refinement operation mainly concerns QA and dialogue data.

\subsection{Data Comparison}\label{sec:data_comparison}
\paragraph{Settings.} We collect the training data of LL3DA \citep{chen2024ll3da} to train \agent and compare the quantitative results with \agent trained with our original data to showcase the impact of training data. We report the performances on Scan2Cap and ScanQA, where their data overlaps ours.

\begin{table}[t!]
\centering
\captionof{table}{Quantitative comparison between \agent trained on the LL3DA data and our data. Metrics follow \cref{tab:vl_results}.}
\resizebox{0.7\linewidth}{!}{
\begin{tabular}{lcccccccccc}
    \toprule
     & \multicolumn{5}{c}{Scan2Cap (val)} & \multicolumn{5}{c}{ScanQA (val)} \\
     \cmidrule(lr){2-6} \cmidrule(lr){7-11}
     & C & B-4 & M & R & Sim & C & B-4 & M & R & EM@1 \\
    \midrule

    LL3DA data & 73.9 & 43.5 & 30.2 & 65.0 & 63.4 & 99.7 & \textbf{14.8} & 19.7 & 47.8 & 22.9 \textcolor{gray}{(46.4)} \\

    Our data & \textbf{86.4} & \textbf{44.4} & \textbf{30.9} & \textbf{65.8} & \textbf{65.6} & \textbf{104.9} & 13.8 & \textbf{20.4} & \textbf{50.3} & \textbf{24.5 \textcolor{gray}{(49.2)}} \\
    \bottomrule
\end{tabular}
}
\label{tab:data_comparison}
\end{table}

The results in \cref{tab:data_comparison} exhibit a consistent performance gap between training on LL3DA data and our original data, underscoring the advantage of our collected training data.

\subsection{Model Comparison}\label{sec:model_comparison}
\paragraph{Settings.} \agent adopts an object-centric 3D representation to encode 3D scenes, which is a novel approach compared with recent works. For example, 3D-LLM \citep{hong20233d} leverages 2D foundation models to obtain dense semantic features and lift them to 3D space, and LL3DA \citep{chen2024ll3da} adopts scene-level encoding. They both use learnable queries to extract 3D features. Here we investigate the influence of model design with the same training data. For a fair comparison, we use Mask3D \citep{schult2022mask3d} object proposals instead of ground-truth masks for the evaluation results of \agent.

\paragraph{LL3DA \vs \agent.} We train \agent on the LL3DA training data and compare the performances with LL3DA generalist results (without task-specific fine-tuning). From the results in \cref{tab:ll3da_leo}, we highlight two takeaways: 1) with the same training data, \agent outperforms LL3DA on most metrics; 2) the gap between LL3DA and \agent is significant on ScanQA, which indicates a major advantage of object-centric 3D representation lies in handling the reasoning task.

\begin{table}[t!]
\centering
\captionof{table}{Quantitative comparison between LL3DA and \agent when both trained on LL3DA data. Metrics follow \cref{tab:vl_results}.}
\resizebox{0.73\linewidth}{!}{
\begin{tabular}{lcccccccccccc}
    \toprule
     & \multicolumn{4}{c}{Scan2Cap (val)} & \multicolumn{4}{c}{Nr3D (val)} & \multicolumn{4}{c}{ScanQA (val)} \\
     \cmidrule(lr){2-5} \cmidrule(lr){6-9} \cmidrule(lr){10-13}
     & C & B-4 & M & R & C & B-4 & M & R & C & B-4 & M & R \\
    \midrule

    LL3DA & 63.0 & 36.0 & 25.7 & 54.7 & \textbf{23.9} & \textbf{13.4} & 22.3 & 45.8 & 75.7 & 13.3 & 15.4 & 37.0 \\

    \agent & \textbf{64.9} & \textbf{37.2} & \textbf{27.4} & \textbf{57.5} & 22.1 & 10.9 & \textbf{22.9} & \textbf{46.3} & \textbf{99.2} & \textbf{14.9} & \textbf{19.4} & \textbf{47.3} \\
    \bottomrule
\end{tabular}
}
\label{tab:ll3da_leo}
\end{table}

\paragraph{3D-LLM \vs \agent.} As LL3DA collects a subset (ScanNet part) of 3D-LLM training data, we leverage this subset to pretrain \agent and compare the downstream performances with 3D-LLM. In contrast to the task-specific fine-tuning results of 3D-LLM, we report \agent's evaluation results after instruction tuning without task-specific fine-tuning. The results in \cref{tab:3dllm_leo} show that \agent consistently outperforms 3D-LLM when adopting the same training data. Notably, the magnitude of this subset is much smaller than their original training data, which further underscores the efficiency of our model.

\begin{table}[t!]
\centering
\captionof{table}{Quantitative comparison between 3D-LLM and \agent when both trained on 3D-LLM data. Metrics follow \cref{tab:vl_results}.}
\resizebox{0.53\linewidth}{!}{
\begin{tabular}{lcccccc}
    \toprule
     & \multicolumn{5}{c}{ScanQA (val)} & SQA3D (test) \\
     \cmidrule(lr){2-6} \cmidrule(lr){7-7}
     & C & B-4 & M & R & EM@1 & EM@1 \\
    \midrule

    3D-LLM & 74.5 & 12.9 & 15.1 & 37.5 & 21.2 & 49.8 \\

    \agent & \textbf{97.4} & \textbf{14.6} & \textbf{19.1} & \textbf{46.8} & \textbf{23.2 \textcolor{gray}{(45.4)}} & \textbf{50.6 \textcolor{gray}{(52.9)}} \\
    \bottomrule
\end{tabular}
}
\label{tab:3dllm_leo}
\end{table}

\subsection{Embodied Acting}\label{sec:result_objnav_additional}

\textbf{Quantitative results of \texttt{ObjNav}.} We provide additional results of \agent 1) generalizing to unseen objects on MP3D (below is a list of the objects used during training ({\color{mygreen}{seen}}) and for OOD evaluation ({\color{myred}{unseen}})), 2) learning with 70K human demonstrations provided by Habitat-web~\citep{ramrakhya2022habitat} instead of shortest path, and 3) learning without one modality (full vs. w/o 3D vs. w/o 2D). Evaluation results are shown in \cref{tab:result_objnav_ood_human}. Note that the baseline Habitat-web is unable to generalize to novel objects as it uses categorical embedding rather than natural language to represent object goals.
\begin{tcolorbox}
\begin{minipage}{\linewidth}
\# \texttt{Objects ({\color{mygreen}{seen}})}\\ \texttt{``gym\_equipment'', ``tv\_monitor'', ``picture'', ``counter'', ``chair'', ``cabinet'', ``table'', ``stool'', ``plant'', ``towel'', ``sofa'', ``cushion'', ``sink'', ``fireplace'', ``toilet'', ``seating'', ``chest\_of\_drawers'', ``bed'', ``shower'', ``bathtub'', ``clothes''} \\
\\
\# \texttt{Objects ({\color{myred}{unseen}})}\\ \texttt{``shelf'', ``pillow'', ``lamp'', ``box'', ``desk'', ``refrigerator'', ``vase'', ``armchair''} 
\end{minipage}
\end{tcolorbox}

\begin{table}[t!]
\centering
\caption{\textbf{Results on object navigation with OOD objects and human demonstrations.} Note that the baseline Habitat-web is unable to generalize to MP3D-{\color{myred}{unseen}} as it uses categorical embedding rather than natural language to represent object goals.}
\small
\setlength\tabcolsep{2pt}
\begin{tabular}{llccccc}
\toprule
 \multicolumn{2}{c}{\multirow{2}{*}{}} & \multicolumn{2}{c}{MP3D-{\color{mygreen}{seen}}} & & \multicolumn{2}{c}{MP3D-{\color{myred}{unseen}}} \\ \cmidrule(lr){3-4}\cmidrule(lr){6-7}
\multicolumn{2}{l}{} &  \small{Success$(\uparrow)$} & \small{SPL$(\uparrow)$} & & \small{Success$(\uparrow)$} &  \small{SPL$(\uparrow)$}\\ \midrule
\multicolumn{2}{l}{Habitat-web (shortest)} & 4.4 & 2.2 & & - &  - \\
\multicolumn{2}{l}{Habitat-web (70k demo)} & \textbf{35.4} & 10.2 & & - & - \\
\midrule
\multicolumn{2}{l}{\agent (shortest, w/o 2D)} & 7.8 & 4.6 & & - & - \\
\multicolumn{2}{l}{\agent (shortest, w/o 3D)} & 8.6 & 6.8 & & - & - \\
\multicolumn{2}{l}{\agent (shortest)} & 23.1 & \textbf{15.2} & & \textbf{11.1} &  \textbf{9.6} \\
\multicolumn{2}{l}{\agent (70k demo)} & 7.1 & 5.3 & & 8.9 & 8.6 \\
\bottomrule
\end{tabular}
\label{tab:result_objnav_ood_human}
\end{table}

The results show that \agent can generalize to novel objects. On the other hand, human demonstrations include more explorations, compared with shortest-path data. Therefore, it will be much harder for agents without a recurrent module (\eg, \agent) to learn from human demonstrations (see \cref{sec:supp_eai_split}), leading to significantly weaker performances. Finally, more modalities (2D + 3D) are indeed crucial to the success of mastering object navigation skills.

\textbf{Qualitative results}.~~We provide more qualitative results of robotic manipulation and embodied navigation in the supplementary video.


\subsection{Scan2Cap}
We provide additional qualitative results on Scan2Cap validation set in \cref{tab:Scan2Cap_add}. The results show that \agent can correctly refer to the queried object and provide accurate descriptions, including spatial relationships with other objects. However, \agent's responses are confined to simple formats that lack diversity. How to unlock more flexible responses while maintaining accuracy can be a direction for future research. 

\begin{table}[t]
    \centering
    \caption{\textbf{Additional qualitative results on Scan2Cap validation set}.}
    \small
    \begin{tabular}{c|p{8em}|p{8em}|p{10em}}
    \toprule
         Scene  & Instruction & Response & Ground Truth  \\
         \midrule
            \raisebox{-1\height}
            {\includegraphics[width=0.2\textwidth]{figs/scan2cap_keyboard.jpg}} & 
            Develop a description of the object at the specified position in the 3D backdrop. & This is a black keyboard. It is in front of two monitors. &
            The keyboard sits atop the workstation desk. It is located in front of the left-side monitor.\\
         \midrule
            \raisebox{-1\height}
            {\includegraphics[width=0.2\textwidth, ]{figs/scan2cap_pillow.jpg}} &
            Focus on the object at the given 3D location and describe it. & This is a white pillow. It is on the couch. & It is a small pillow located on the couch.  You can notice it directly on your left when walking through the door into the room. \\
         \midrule
         \raisebox{-1\height}
         {\includegraphics[width=0.2\textwidth, ]{figs/scan2cap_cabinet.jpg}} & Provide a clearer description of the object at that specific 3D spot. &  This is a white cabinet. It is to the right of a couch. &
         This is a white ottoman. It is located to the right of the black couch. \\
        \midrule
            \raisebox{-1\height}
            {\includegraphics[width=0.2\textwidth]{figs/scan2cap_door.jpg}} &
            What does the object at the specified position in the 3D visualization look like? & There is a rectangular brown door. It is next to a couch. & There is a door in the middle of the northern wall next to a metal cabinet. The door is shaped like a brown rectangle and as a dark yellow colored doorknob. \\
        \bottomrule
\end{tabular}

\label{tab:Scan2Cap_add}
\end{table}

\subsection{ScanQA}

We provide additional qualitative results on ScanQA validation set in \cref{tab:ScanQA_add} and categorize the responses into several types:
\begin{itemize}[leftmargin=*]
    \item \textit{Wrong}. The response is inaccurate and deemed wrong.
    \item \textit{Wrong but reasonable}. The response is deemed wrong but is reasonable to some extent, probably due to ambiguities in the scene. Consider the second case in \cref{tab:ScanQA_add}. There are many objects such as a coat rack, a coat, and a mini fridge-shaped cabinet on the right side of the organizer. Though \agent's response ``mini fridge'' does not match the ground truth ``coat rack'', it is consistent with the 3D scene layout.
    \item \textit{Wrong but accurate}. The response is accurate according to the scene but is deemed wrong due to imperfect ground truth annotations.
    \item \textit{Correct}. The response is accurate and deemed correct.
    \item \textit{Correct and more accurate}. The response is more accurate than the ground truth annotations.
\end{itemize}
 


\begin{table}[t]
    \centering
    \caption{\textbf{Additional qualitative results on ScanQA validation set}.}
    \small
    \begin{tabular}{c|p{6em}|p{6em}|p{8em}|p{6em}}
    \toprule
         Scene  & Question & Response & Ground Truth & Response Type \\
         
         \midrule
         \raisebox{-1\height}
         {\includegraphics[width=0.2\textwidth, ]{figs/vis_scanqa_2.jpg}} & What is placed on the black chair out of the three chairs present in the room? & pillow & tan jacket, guitar box like object & \textit{Wrong} \\
         
         \midrule
         \raisebox{-1\height}
         {\includegraphics[width=0.2\textwidth, ]{figs/vis_scanqa_1.jpg}} & What is next to the organizer shelf? & mini fridge & coat rack & \textit{Wrong but reasonable} \\
         
         \midrule
         \raisebox{-1\height}
         {\includegraphics[width=0.2\textwidth, ]{figs/vis_scanqa_3.jpg}} & Where is the dresser located? &  to right of door &
         underneath television next to black guitar case and green guitar case, under tv set next to guitar cases & \textit{Wrong but accurate} \\

         \midrule
         \raisebox{-1\height}
         {\includegraphics[width=0.2\textwidth, ]{figs/vis_scanqa_4.jpg}} &
         What kind of desk is the electric piano on? &
         wooden desk & wooden desk & \textit{Correct} \\
        
         \midrule
         \raisebox{-1\height}
         {\includegraphics[width=0.2\textwidth]{figs/vis_scanqa_4.jpg}} &
         How many monitor sits on top of the desk? & 2 & 2 & \textit{Correct} \\
         
         \midrule
            \raisebox{-1\height}
            {\includegraphics[width=0.2\textwidth]{figs/vis_scanqa_6.jpg}} & 
            What is on the left side of the brown desk? &
            black computer monitor & monitor / computer monitor &
            \textit{Correct and more accurate} \\
        \bottomrule
\end{tabular}

\label{tab:ScanQA_add}
\end{table}

\subsection{SQA3D}
We provide additional qualitative results on SQA3D test set in \cref{tab:sqa_add} and follow the aforementioned response types. The embodied reasoning in SQA3D requires the understanding of not only the scene but also the situation of embodiment. In \cref{tab:sqa_add}, answering ``What am I sitting at?'' necessitates that \agent accurately identifies the objects at its current location. And the response to ``How many beds are in front of me?'' indicates that \agent can reason based on the understanding of its orientation.

\begin{table}[t!]
    \centering
    \caption{\textbf{Additional qualitative results on SQA3D test set}.}
    \small
    \begin{tabular}{c|p{6em}|p{6em}|p{8em}|p{6em}}
        \toprule
         Scene & Question & Response & Ground Truth & Response Type \\
         \midrule
         \raisebox{-1\height}
         {\includegraphics[width=0.2\textwidth]{figs/vis_sqa_2.jpg}} & What is the only orange object on my left side? & backpack & jacket & \textit{Wrong} \\
         \midrule
         \raisebox{-1\height}
         {\includegraphics[width=0.2\textwidth]{figs/vis_sqa_4.jpg}} & Is there anything on the desk? & yes & laptop & \textit{Wrong but accurate} \\
         \midrule
         \raisebox{-1\height}
         {\includegraphics[width=0.2\textwidth,]{figs/vis_sqa_1.jpg}} & What type of bed is the one on the right? & loft bed & loft bed & \textit{Correct} \\
         \midrule
         \raisebox{-1\height}
         {\includegraphics[width=0.2\textwidth]{figs/vis_sqa_4.jpg}} & What am I sitting at? & desk & desk & \textit{Correct} \\
         \midrule
         \raisebox{-1\height}
         {\includegraphics[width=0.2\textwidth]{figs/vis_sqa_5.jpg}} & How many beds are in front of me? & one & one & \textit{Correct} \\
         \bottomrule
    \end{tabular}
    
    \label{tab:sqa_add}
\end{table}

\begin{table}[t!]
    \centering
    \caption{\textbf{Examples from our datasets}. We provide the tasks, 2D images (optional), 3D point clouds, and instructions as the input of \agent, as well as the expected responses of \agent.}
    \small
    \begin{tabular}{c|c|p{6em}|p{12em}}
    \toprule
         Dataset \& Task & 2D \& 3D input & Instruction & Response  \\
         \midrule
         \makecell{\agent-align\\object captioning} & \raisebox{-1\height}{\includegraphics[width=0.2\textwidth, ]{figs/cap3d_data.jpg}} & USER: Explain the object at the pinpointed location in the 3D setting. ASSISTANT: & a yellow metal cage with a fence around it. \\
         \midrule
         \makecell{\agent-align\\object referring} & \raisebox{-1\height}{\includegraphics[width=0.2\textwidth, ]{figs/obj_refer_3d.jpg}}& USER: Give me a snapshot of the object at the chosen 3D location. ASSISTANT:& find the office chair that is near the copier\\
         \midrule
         \makecell{\agent-align\\scene captioning} & \raisebox{-1\height}{\includegraphics[width=0.2\textwidth, ]{figs/scene_cap_data.png}}& USER: Relay a brief summary of this scene's elements. ASSISTANT: &
         In this scene, there is a gray flat picture hanging on the wall, standing in front of a windowsill. There is also a rectangular white padded pillow lying on the bed, close to a big rectangular white padded pillow. The room has a white architectural wall and a flat overhead ceiling. The bed is tidy and white, located in front of a narrow tall commode. The curtains are hanging on the wall. The room has a calm and comfortable atmosphere, suitable for relaxation and sleep.\\ %
        \bottomrule
\end{tabular}

\label{tab:supp_data_example}
\end{table}

\begin{table}[t!]
\centering
\caption{\textbf{Examples from our datasets}. Continued.}
\small
\begin{tabular}{c|c|p{8em}|p{10em}}
\toprule     
         \makecell{\agent-instruct\\3D captioning} & \raisebox{-1\height}{\includegraphics[width=0.2\textwidth, ]{figs/scan2cap_door.jpg}}& USER: What does the object at the specified position in the 3D visualization look like? ASSISTANT:& there is a door in the middle of the northern wall next to a metal cabinet. the door is shaped like a brown rectangle and as a dark yellow colored doorknob.\\
         \midrule
         \makecell{\agent-instruct\\3D QA} & \raisebox{-1\height}{\includegraphics[width=0.2\textwidth, ]{figs/qa_caption_3d.jpg}}& USER: What is in the right corner of room by curtains? ASSISTANT: & brown cabinet with tv sitting in it\\
         \midrule
         \makecell{\agent-instruct\\3D dialogue} &\raisebox{-1\height}{\includegraphics[width=0.2\textwidth, ]{figs/3rdialog_data.jpg}} & USER: I'm redecorating my living room. Can you give me some suggestions on how to arrange the furniture?
         
         ASSISTANT: Of course! I'd be happy to help. What furniture do you have in your living room?
         
         USER: I have a sofa, a table, a chair, and a couch table. ASSISTANT: &
         Great! Here's a suggestion: you can place the sofa against the wall, with the table in front of it. The chair can be placed to the right of the table, and the couch table can be placed to the right of the chair. This arrangement creates a cozy seating area. \\% 3rscan 95be45dd-a558-22da-9de4-002d61e13deb  
         \midrule
         \makecell{\agent-instruct\\task planning} & \raisebox{-1\height}{\includegraphics[width=0.2\textwidth, ]{figs/3rplan_data.jpg}} & USER: Develop a plan for this: Arrange and embellish the family room. ASSISTANT: &
         1. Arrange and fluff the cushions and pillows on the armchair.
         
         2. Place decorations, such as figurines or vases, on the cabinet or shelf.
         
         3. Place hygiene products, such as wipes or lotions, on the commode.
         
         4. Turn on the lights and adjust their brightness if needed.
         
         5. Arrange diapers in a designated storage area.
         
         6. Place a blanket or pillow on the armchair for added comfort.
         
         7. Adjust the festoon or other decorative elements if desired.\\ %
    \bottomrule
    \end{tabular}
    
    \label{tab:supp_data_example_cont}
\end{table}

\begin{table}[t!]
\centering
\caption{\textbf{Examples from our datasets}. Continued.}
\small
\begin{tabular}{c|c|p{8em}|p{12em}}
\toprule     
         \makecell{\agent-instruct\\embodied navigation} &
         \raisebox{-1\height}{\makecell{\includegraphics[width=0.2\textwidth, ]{figs/mp3d_objnav_2d.png} \\ \includegraphics[width=0.2\textwidth, ]{figs/mp3d_objnav_3d.jpg}}}& USER: The task is navigation. Your goal is to find counter by moving around in the scene. Past actions: <31999> <31999> <31999> <31999>. ASSISTANT:& <31996>\\
         \midrule 
         \makecell{\agent-instruct\\robotic manipulation} & \raisebox{-1\height}{\makecell{\includegraphics[width=0.2\textwidth, ]{figs/cliport_2d.png}\\\includegraphics[width=0.2\textwidth, ]{figs/cliport_3d.jpg}}}& USER: The task is manipulation. Your goal is to put the blue blocks in a green bowl. Past actions: <31991> <31671> <31511> <31991> <31671> <31511> <31991> <31671> <31511> <31991> <31671> <31511> <31991> <31671> <31511> <31991> <31671> <31511> <31991> <31671> <31511> <31991> <31671> <31511>. ASSISTANT:& <31748> <31644> <31511> <31736> <31595> <31500> \\     
    \bottomrule
    \end{tabular}
    
    \label{tab:supp_data_example_cont2}
\end{table}


\newpage
\section*{NeurIPS Paper Checklist}
\label{sec:check_list}
\input{check_list}

\end{document}