{\color{warningcolor} In this subsection, all the following content comes from CycleResearcher-12B and CycleReviewer-123B.}

\textcolor[RGB]{0,0,139}{Below is a generation example from the CycleResearcher-12B model, with all experiments being genuine and valid. Specifically, we first used the CycleResearcher-12B model to generate the motivation, main idea, paper title, abstract, introduction, and methodology. The model then conducted detailed experimental planning, generating six different experimental groups. Building on this, we used GPT-01-preview model with AI Scientist as the baseline to generate code for these experiments, costing approximately \$20 and taking 6 hours on a single A100 GPU server. After obtaining the experimental results, we compiled all results and experimental figures into a JSON file and input it back into the CycleResearcher-12B model. Finally, the CycleResearcher-12B model automatically analyzed the experimental results and wrote the remaining sections of the paper, including experimental analysis, related work, experimental conclusions, and ethical statements.}

\textcolor[RGB]{0,0,139}{Throughout the process, we employed CycleResearcher as the thinker, responsible for reading literature, contemplating the research process, and writing experimental reports. GPT-01-preview served as the executor, responsible for implementing the experimental setups planned by CycleResearcher step by step.}

\subsubsection{Outline}
\begin{example}{}{test3}
\textcolor[RGB]{0,0,139}{The increasing scale of deep neural networks has led to a diverse range of behaviors, some of which are predictable, like the improvement in predictive ability with more data, and others are surprising, like grokking and emergent abilities. Understanding these phenomena is crucial for anticipating and steering the impact of increasingly powerful AI systems. Grokking, a phenomenon where overfitting is followed by generalization, has been studied by various works but often in different settings, making it difficult to establish a unified understanding. Emergent abilities, where behaviors appear only at scale, are also important to study. However, previous works have focused on language models, leaving a gap in understanding grokking and emergent abilities in other settings. This paper aims to bridge this gap by studying grokking and emergent abilities in the context of neural networks trained on synthetic algorithmic tasks. The goal is to provide a clear framework for understanding these phenomena and to identify the underlying mechanisms that drive them.}

\end{example}
\begin{note}{}{test3}
\textcolor[RGB]{0,0,139}{The paper proposes the 'generalization gap' as a way to understand grokking and emergent abilities in neural networks. It defines the generalization gap as the difference in loss on the training and test sets and shows that these phenomena can be observed in simple synthetic algorithmic tasks. The paper introduces four measures of the generalization gap—peakness, inflection point, area of inflection, and length of inflection—to characterize different phases of training. Based on these measures, the paper hypothesizes that grokking and emergent abilities occur when the generalization gap takes a certain form. The hypothesis is validated through experiments on neural networks with varying architecture, parameterization, and training data. The paper also explores the relationship between grokking and double descent, finding that emergent abilities can be seen as a form of grokking, with the two phenomena sharing the same mathematical form.}

\end{note}
